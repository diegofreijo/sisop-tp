\subsection{Ejercicio 3: Comandos b'asicos MINIX/UNIX}
\subsubsection{Pongale password root a root.}

Utilizando el comando passwd:

\begin{verbatim}
    # passwd
    Changing the shadow password of root
    New password:
    Retype password:
    # exit
    Minix Release 2.0 Version 0
    noname login: root
    Password:
    #
\end{verbatim}

\subsubsection{pwd}

Indique qu� directorio pasa a ser su $current\ directory$ si ejecuta:

\subsubsubsection{1.3.2.1. $\sharp$ cd /usr/src}

\begin{verbatim}
    # cd /usr/src
    # pwd
    /usr/src
\end{verbatim}

Utilizando el comando \textbf{pwd} podemos ver en que directorio nos encontramos. En 'este caso el $current\ directory$ es \textbf{/usr/src}.

\subsubsubsection{1.3.2.2. $\sharp$ cd}

\begin{verbatim}
    # cd
    # pwd
    /
    #
\end{verbatim}

Nuevamente utilizamos el comando \textbf{pwd} y el directorio actual es \textbf{/}.

\subsubsubsection{1.3.2.3. �C'omo explica el punto 1.3.2.2.?}

Ejecutar el comando \textbf{cd} sin argumentos cambia el directorio actual al directorio home del usuario. Para el usuario root 'este directorio es el directorio ra'iz del sistema.

\subsubsection{cat}

Cu'al es el contenido del archivo \textbf{/usr/src/.profile} y para qu'e sirve.

\begin{verbatim}
    # cat /usr/src/.profile
    # Login shell profile.
    
    # Environment.
    umask 022
    PATH=/usr/local/bin:/bin:/usr/bin
    PS1="! "
    export PATH
    
    # Erase character, erase line, and interrupt keys.
    stty erase '^H' kill '^U' intr '^?'
    # Check terminal type.
    case $TERM in
    dialup|unknown|network)
            echo -n "Terminal type? ($TERM) "; read term
            TERM="${term:-$TERM}"
    esac
    
    # Shell configuration.
    case "$0" in *ash) . $HOME/.ashrc;; esac
    #
\end{verbatim}

Cada usuario tendr'a 'este script en su home, y cada vez que se "loguee" se ejecutar'a.

\subsubsection{find}

En qu'e directorio se encuentra el archivo \textbf{proc.c}

\begin{verbatim}
    # find / -name proc.c
    /usr/src/kernel/proc.c
\end{verbatim}

\subsubsection{mkdir}

Genere un directorio \textbf{/usr/$<$nombregrupo$>$}

\begin{verbatim}
    #mkdir /usr/grupo8
    # cd /usr
    # ls
    adm bin grupo8  lib   man  preserve src
    ast etc include local mdec spool    tmp
\end{verbatim}

\subsubsection{cp}

Copie el archivo \textbf{/etc/passwd} al directorio \textbf{/usr/$<$nombregrupo$>$}

\begin{verbatim}
    # cp /etc/passwd /usr/grupo8/passwd
    # cd /usr/grupo8
    # ls
    passwd
    #
\end{verbatim}

\subsubsection{chgrp}

Cambie el grupo del archivo \textbf{/usr/$<$grupo$>$/passwd} para que sea other.
\begin{verbatim}
    # pwd
    /usr/grupo8
    # ls -l
    total 1
    -rw-r--r-- 1 root operator 285 Jun 17 22:05 passwd
    # chgrp other passwd
    # ls -l
    total 1
    -rw-r--r-- 1 root other 285 Jun 17 22:05 passwd
    #
\end{verbatim}

\subsubsection{chown}

Cambie el propietario del archivo \textbf{/usr/$<$grupo$>$/passwd} para que sea \textbf{ast}.

\begin{verbatim}
    # chown ast passwd
    # ls -l
    total 1
    -rw-r--r-- 1 ast other 285 Jun 17 22:06 passwd
    #
\end{verbatim}

\subsubsection{chmod}

Cambie los permisos del archivo \textbf{/usr/$<$grupo$>$/passwd} para que:

\begin{itemize}
\item el propietario tenga permisos de lectura, escritura y ejecuci'on
\item el grupo tenga solo permisos de lectura y ejecuci'on
\item el resto tenga solo permisos de ejecuci'on
\end{itemize}

\begin{verbatim}
    # chmod 751 passwd
    # ls -l
    total 1
    -rwxr-x--x 1 ast other 285 Jun 17 22:11 passwd
    #
\end{verbatim}

\subsubsection{grep}

Muestre las lineas que tiene el texto \textbf{include} en el archivo \textbf{/usr/src/kernel/main.c}

\begin{verbatim}
    # grep include /usr/src/kernel/main.c
    #include "kernel.h"
    #include <signal.h>
    #include <unistd.h>
    #include <minix/callnr.h>
    #include <minix/com.h>
    #include "proc.h"
    #
\end{verbatim}

Muestre las lineas que tiene el texto \textbf{POSIX} que se encuentren en todos los archivos \textbf{/usr/src/kernel/}

\begin{verbatim}
# grep POSIX /usr/src/kernel/*
/usr/src/kernel/kernel.h:#define _POSIX_SOURCE 1 /* tell headers to inclu
de POSIX stuff */
/usr/src/kernel/rs232.c: if ((tp->tty_termios.c_lflag & IXON) && rs->oxoff !=
_POSIX_VDISABLE)
/usr/src/kernel/system.c: * SYS_SENDSIG send a signal to a process (POSIX style)
/usr/src/kernel/system.c: * SYS_SIGRETURN complete POSIX-style signalling
/usr/src/kernel/system.c:/* Handle sys_sendsig, POSIX-style signal */
/usr/src/kernel/system.c:/* POSIX style signals require sys_sigreturn to put
things in order before the
/usr/src/kernel/tty.c:/* These Posix functions are allowed to fail if _POSIX_JOB
_CONTROL is
/usr/src/kernel/tty.c: /* _POSIX_VDISABLE is a normal character value, so better
escape it. */
/usr/src/kernel/tty.c: if (ch == _POSIX_VDISABLE) ch |= IN_ESC;
#
\end{verbatim}

\subsubsection{su}

\subsubsubsection{1.3.11.1. Para qu'e sirve?}

Permite convertir un usuario en otro sin tener que desconectarse del sistema.
Por defecto lo convierte en superuser. Le pedir'a la password correspondiente.

\subsubsubsection{1.3.11.2. Qu'e sucede si ejecuta el comando \textbf{su} estando logueado como \textbf{root}?}

Se genera un shell nuevo. Como estabamos logueados como root, no pidi'o password.

\subsubsubsection{1.3.11.3. Genere una cuenta de $<$usuario$>$}
%el usuario es Maxi pero me pareci� mejor poner pepe

\begin{verbatim}
    # adduser pepe other /usr/pepe
    cpdir /usr/ast /usr/pepe
    chown -R 10:3 /usr/pepe
    echo pepe::0:0::: >>/etc/shadow
    echo pepe:##pepe:10:3:pepe:/usr/pepe: >>/etc/passwd
    The new user pepe has been added to the system. Note that the password,
    full name, and shell may be changed with the commands passwd(1), chfn(1),
    and chsh(1). The password is now empty, so only console logins are possible.
\end{verbatim}

\subsubsubsection{1.3.11.4. Entre a la cuenta $<$usuario$>$ generada}

\begin{verbatim}
    Minix Release 2.0 Version 0
    noname login: pepe
    $
\end{verbatim}

No pidi'o Password pues el usuario pepe no tiene a'un una seteada.
 
\subsubsubsection{1.3.11.5. Repita los comandos de 1.3.11.2}

Estando logueado como pepe ejecutamos el comando \textbf{su}.

\begin{verbatim}
    $ su
    Password:
    #
\end{verbatim}

Nos pide la password de root pues el usuario pepe es parte del grupo other. Si fuera parte del grupo operator no la pedir'ia.

\subsubsection{passwd}

\subsubsubsection{1.3.12.1. Cambie la password del usuario nobody}

\begin{verbatim}
    # passwd nobody
    Changing the password of nobody
    New password:
    Retype password:
    #
\end{verbatim}

\subsubsubsection{1.3.12.2. Presione las teclas ALT-F2 y ver'a otra sesi'on MINIX. Loguearse como nobody }

\begin{verbatim}
    Minix Release 2.0 Version 0
    noname login: nobody
    Password:
    $
\end{verbatim}

Nos logueamos con password nobody.

\subsubsubsection{1.3.12.3. Ejecutar el comando su}

\begin{verbatim}
    $ su
    Password:
    #
\end{verbatim}

\subsubsubsection{1.3.12.3.1. Qu'e le solicita?}

Solicita la password de root.

\subsubsubsection{1.3.12.3.2. Sucede lo mismo que en 1.3.11.2?�Por qu'e?}

No, cuando se realiz'o lo mismo con el usuario \textbf{root} no fue necesario ingresar la password. Esto se debe a que el usuario \textbf{nobody} pertenece al grupo \textbf{nogroup}, quienes no tienen permisos para loguearse como \textbf{root} y, de hecho, casi no poseen ningun tipo de privilegios.

\subsubsection{rm}

Suprima el archivo \textbf{/usr/$<$grupo$>$/passwd}.
\begin{verbatim}
    # ls /usr/grupo8
    passwd
    # rm /usr/grupo8/passwd
    # ls /usr/grupo8
    #
\end{verbatim}

\subsubsection{ln}

Enlazar el archivo \textbf{/etc/passwd} a los siguientes archivos \textbf{/tmp/contra1} \textbf{/tmp/contra2}.
Hacer un ls -l para ver cuantos enlaces tiene \textbf{/etc/passwd}.

\begin{verbatim}
    # ls -l /etc/passwd
    -rw-r--r-- 1 root operator 314 Apr 15 2007 /etc/passwd
    # ln /etc/passwd /tmp/contra1
    # ln /etc/passwd /tmp/contra2
    # ls -l /etc/passwd
    -rw-r--r-- 3 root operator 314 Apr 15 2007 /etc/passwd
    #
\end{verbatim}

En el listado del archivo, la segunda columna (luego de los permisos) indica la cantidad de links que tiene ese archivo.

\subsubsection{mkfs}

Genere un Filesystem MINIX en un diskette

\begin{verbatim}
    # mkfs /dev/fd0
\end{verbatim}

\subsubsection{mount}

\subsubsubsection{M'ontelo en el directorio \textbf{/mnt}.}

\begin{verbatim}
    # mount /dev/fd0 /mnt
    /dev/fd0 is read-write mounted on /mnt
    #
\end{verbatim}

\subsubsubsection{Presente los filesystems que tiene montados.}

\begin{verbatim}
    # mount
    /dev/hd1 is root device
    /dev/hd2 is mounted on /usr
    /dev/fd0 is mounted on /mnt
    #
\end{verbatim}


\subsubsection{df}

\subsubsubsection{Qu'e espacio libre y ocupado tienen todos los filesystems montados?(en KBYTES)}

\begin{verbatim}
# df
Device    Inodes Inodes Inodes  Blocks Blocks Blocks  Mounted   V Pr
           total  used   free    total  used   free      on         
           -----  -----  -----   -----  -----  -----    ------   - --
/dev/hd1    496    212    284     1480    700    780    /        2 rw
/dev/hd2  12528   3309   9219    75096  27443   47653   /usr     2 rw
/dev/fd0    480      1    479     1440     35    1405   /mnt     2 rw
#
\end{verbatim}

En todos los dispositivos, el tama'no de bloque es 1KB, por lo tanto el espacio disponible en KB es el indicado por la columna \textbf{Blocks free} y el espacio ocupado es el indicado por la columna \textbf{Blocks used}.

\subsubsection{ps}

\subsubsubsection{1.3.18.1. �Cu�ntos procesos de usuario tiene ejecutando?}

\begin{verbatim}
    # ps -a
    PID TTY  TIME CMD
     41  co  0:00 -sh
     42  c1  0:00 getty
     47  c0  0:00 ash
     48  c0  0:00 ps -a
\end{verbatim}

Los procesos de usuario que se est'an ejecutando son 4 (incluyendo al \textbf{ps}).

\subsubsubsection{1.3.18.2. Indique cu'antos son del sistema}

\begin{verbatim}
    # ps -ax
    PID TTY  TIME CMD
      0   ?  0:00 TTY
      0   ?  0:00 SCSI
      0   ?  0:00 WINCH
      0   ?  0:00 SYN_AL
      0   ?  8:35 IDLE
      0   ?  0:00 PRINTER
      0   ?  0:00 FLOPPY
      0   ?  0:00 MEMORY
      0   ?  0:00 CLOCK
      0   ?  0:01 SYS
      0   ?  0:06 HARDWAR
      0   ?  0:00 MM
      0   ?  0:04 FS
      1   ?  0:00 INIT
     40  co  0:00 -sh
     27   ?  0:00 update
     41  c1  0:00 getty
     47  co  0:00 ash
     57  co  0:00 ps -ax
    #
\end{verbatim}

Aqu'i se listan todos los procesos incluyendo tambi'en a los de usuario. En total hay 18 procesos ejecut'andose, 4 son de usuario y 15 del sistema.

\subsubsection{umount}

\subsubsubsection{1.3.19.1. Desmonte el Filesystem del directorio /mnt}

\begin{verbatim}
    # umount /dev/fd0
    /dev/fd0 unmounted from /mnt
    #
\end{verbatim}

\subsubsubsection{1.3.19.2. Monte el Filesystem del diskette como read-only en el directorio /mnt}

\begin{verbatim}
    # mount /dev/fd0 /mnt -r
    /dev/fd0 is read-only mounted on /mnt
    #
\end{verbatim}

\subsubsubsection{1.3.19.3. Desmonte el Filesystem del directorio /mnt}

\begin{verbatim}
    # umount /dev/fd0
    /dev/fd0 unmounted from /mnt
    #
\end{verbatim}

\subsubsection{fsck}

\subsubsubsection{Chequee la consistencia de Filesystem del diskette}

\begin{verbatim}
    # fsck /dev/fd0

    Checking zone map
    Checking inode map
    Checking inode list
    
    blocksize =  1024        zonesize  =  1024
    
         0    Regular files
         1    Directory
         0    Block special files
         0    Character special files
       479    Free inodes
         0    Named pipes
         0    Symbolic links
      1405    Free zones
    #
\end{verbatim}

\subsubsection{dosdir}

\subsubsubsection{Tome un diskette formateado en DOS con archivos y ejecute $dosdir a$}

\begin{verbatim}
    #dosdir a
    dosdir: cannot open /dev/dosA: no such file or directory
    #
\end{verbatim}

\subsubsubsection{Ejecute los comandos necesarios para que funcione correctamente el comando anterior}
El diskette con formato DOS est'a en \textbf{/dev/fd1}

\begin{verbatim}
    #dosdir fd1
    HOLADOS.TXT
    ABMUSR
    GETNEXTG
    MACHAQUE
    #
\end{verbatim}

\subsubsection{dosread}

\subsubsubsection{Copie un archivo de texto desde un diskette DOS al directorio \textbf{/tmp}}

\begin{verbatim}
    #dosread fd1 HOLADOS.TXT
    hola mundo!

    #dosread -a fd1 HOLADOS.TXT > /tmp/holaminix.txt
    #cat /tmp/holaminix.txt
    hola mundo!

    #
\end{verbatim}

\subsubsection{doswrite}

\subsubsubsection{Copie el archivo \textbf{/etc/passwd} al diskette DOS}

\begin{verbatim}
    #doswrite -a fd1 passwd < /etc/passwd
    #dosdir fd1
    HOLADOS.TXT
    ABMUSR
    GETNEXTG
    MACHAQUE
    PASSWD
    #
\end{verbatim}
