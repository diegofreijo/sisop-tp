\subsection{Ejercicio 7}
\subsubsection{Inciso a}
Ejecutar

\verb�# make install�
    
Se compilan todo los archivos del kernel, mm o fs que no tengan su archivo .o generado.

\begin{verbatim}
cd ../kernel && exec make - install
cd keymaps && make - install
make: 'install' is up to date
cd ../mm && exec make - install
make: 'install' is up to date
cd ../fs && exec make - install
make: 'install' is up to date
cd ../inet && exec make - install
make: 'install' is up to date
\end{verbatim}

Ejecutar

\verb�# make hdboot�

Para generar una nueva imagen de Minix

\begin{verbatim}
cd ../kernel && exec make -
make: 'kernel' is up to date
cd ../mm && exec make -
exec cc -c -I/usr/include main.c
exec cc -c -I/usr/include forkexit.c
exec cc -c -I/usr/include break.c
exec cc -c -I/usr/include exec.c
exec cc -c -I/usr/include signal.c
exec cc -c -I/usr/include alloc.c
exec cc -c -I/usr/include utility.c
exec cc -c -I/usr/include table.c
exec cc -c -I/usr/include putk.c
exec cc -c -I/usr/include trace.c
exec cc -c -I/usr/include getset.c
exec cc -c -I/usr/include newcall.c
exec cc -c -I/usr/include llamsistema.c
exec cc -c -I/usr/include semaf.c
exec cc -o mm -i main.o forkexit.o break.o exec.o \
	signal.o alloc.o utility.o table.o \
	putk.o trace.o getset.o newcall.o \
	llamsistema.o semaf.o
install -S 256w mm
cd ../fs && exec make -
exec cc -c -I/usr/include main.c
exec cc -c -I/usr/include open.c
exec cc -c -I/usr/include read.c
exec cc -c -I/usr/include write.c
exec cc -c -I/usr/include pipe.c
exec cc -c -I/usr/include device.c
exec cc -c -I/usr/include path.c
exec cc -c -I/usr/include mount.c
exec cc -c -I/usr/include link.c
exec cc -c -I/usr/include super.c
exec cc -c -I/usr/include inode.c
exec cc -c -I/usr/include cache.c
exec cc -c -I/usr/include cache2.c
exec cc -c -I/usr/include filedes.c
exec cc -c -I/usr/include stadir.c
exec cc -c -I/usr/include protect.c
exec cc -c -I/usr/include time.c
exec cc -c -I/usr/include misc.c
exec cc -c -I/usr/include utility.c
exec cc -c -I/usr/include table.c
exec cc -c -I/usr/include putk.c
exec cc -c -I/usr/include othercall.c
exec cc -o fs -i main.o open.o read.o write.o pipe.o \
	device.o path.o mount.o link.o super.o inode.o \
	cache.o cache2.o filedes.o stadir.o protect.o time.o \
	lock.c misc.o utility.o table.o putk.o othercall.o
install -S 512w fs
installboot -image image ../kernel/kernel ../mm/mm ../fs/fs  init
    text    data     bss     size
   53248    8964   40076   102288  ../kernel/kernel
   14096    1244   30736    46076  ../mm/mm
   29184    2228  108084   139496  ../fs/fs
    6828    2032    1356    10216  init
  ------  ------  ------  -------
  103356   14468  180252   298076  total
exec sh mkboot hdboot
rm /dev/hd1:/minix/2.0.2SISTEMASO
cp image /dev/hd1:/minix/2.0.2SISTEMASOPERATIVOS-UBA-FCENr1
Done.
\end{verbatim}


Ejecutar

\verb�# reboot�

 para reiniciar con la nueva version de kernel

\subsubsection{Inciso a}
Se crea directorio \verb�/minix2� y se copia desde \verb�/minix� el archivo 2.0.0. Al iniciar minix presionar ESC. 
Escribir 

\verb�> image=minix�

Si se quiere iniciar con la nueva versi�n del kernel
	
Escribir 	

\verb�>image= minix2�

Si se quiere iniciar con la versi�n original.

\verb�>boot�

Para iniciar

