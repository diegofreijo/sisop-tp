\subsection{Ejercicio 11}
Se realizaron los siguientes cambios

\verb�/usr/include/minix/callnr.h�

Se increment'o en 1 el define NCALLS
Se agreg� \verb0#define LLAMSISTEMA0 con el nro correspondiente

\verb�/usr/src/mm/proto.h�
Se agrega linea:

\verb�_PROTOTYPE( int do_llamsistema, (void)	);�


\verb�/usr/src/mm/table.c�
Se agrega linea:

\verb�do_llamsistema, 	/\textasteriskcentered 79 = LLAMSISTEMA \textasteriskcentered/�


\verb�/usr/src/fs/table.c�
Se agrega linea:

\verb�no_sys, 		/\textasteriskcentered 79 = LLAMSISTEMA \textasteriskcentered/�


Se implementa System Call en:
\verb�/usr/src/mm/llamsistema.c�
Se actualiza Makefile de \verb�/usr/src/mm� para incluirlo.

Se agrega la linea al archivo
\verb�/usr/include/minix/com.h�

\verb�#define OPC_NEWCALL		mi_i1�

Para renombrar el par'ametro que se usa para la elecci'on de la consulta que se hace por medio del System Call.

Se recompila kernel

Ejecutar

\begin{verbatim}
# /usr/src/tools/make install
# /usr/src/tools/make hdboot
\end{verbatim}



\emph{Nota: no se us'o el nombre pedido x la c'atedra (newcall), ya que este fue usado para otras pruebas. Se realiz'o una version del getpid desde MM y desde FS, sendas llamadas a sistema con nombres newcall y othercall respectivamente.}

\subsubsection{Pruebas}

En imagen minix para ejercicio 11

Fuente: \verb0/usr/ej11/llam.c0

Ejecutable: \verb0/usr/ej11/llam0

Modo de prueba

\begin{verbatim}
./llam [opcion]
\end{verbatim}

Opcion es del 1 al 5

\begin{enumerate}
\item Entrega el pid del programa
\item Entrega el pid del padre, que, al estar implementado en MM, es el pid del mm o sea 0
\item Entrega el puntero al segmento text, en hexadecimal
\item Entrega el puntero al segmento data, en hexadecimal
\item Entrega el puntero al segmento stack, en hexadecimal
\end{enumerate}

Para las opciones 3, 4 y 5, para verificar la correctitud de la misma se agreg'o un ciclo while para evitar que el programa termine. Al ejecutar el programa con algunas de estas opciones, al presionar F2 veremos la informacion de los segmentos correspondientes. Para matar al proceso, abrimos otra consola, ejecutamos ps para ver el nro del mismo y con el comando \verb0kill [nro proc]0 lo eliminamos.

En el mismo directorio el newcall.c y el othercall.c son source para probar dichas funciones
\begin{verbatim}
./newcall
./othercall
\end{verbatim}

Tienen la misma funcionalidad que \verb0./llam 10
La funci'on \verb0NEWCALL0 utiliza la informaci'on del Memory Manager (igual que \verb0LLAMSISTEMA0) para averiguar el pid y \verb0OTHERCALL0, el File System.
