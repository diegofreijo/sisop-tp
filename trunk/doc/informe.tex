\documentclass[spanish, a4paper, 11pt]{article}

\usepackage[a4paper,margin=3.5cm,top=3.0cm,bottom=3.0cm]{geometry}	% Define los margenes
\usepackage[spanish,activeacute]{babel}								% Idioma castellano
\usepackage{caratula}														% Caratula de Algo2
%\usepackage[a4paper=true,pagebackref=true]{hyperref}				% Agrega la TOC al PDF e hipervinculos
\usepackage[pdftex]{graphicx} 											% Permite insertar graficos
\usepackage{fancyhdr}														% Permite manejo de cabeceras de pagina
\usepackage{eufrak}															% Usado en el enunciado del trabajo
\usepackage{latexsym}
%\usepackage{algorithmic}													% Para escribir los algos
%\usepackage{dsfont}															% Para el simbolo de naturales
%\usepackage[font=small,labelfont=bf]{caption}						% Para editar las captions


% Estilo de pagina para tener las cabeceras
\pagestyle{fancy}
\lhead{Sistemas Operativos}
\rhead{Diego Freijo - Maximiliano Giusto - Ignacio Iacobacci}

% Numeracion de paginas
\pagenumbering{arabic}
\parskip=1.5ex

% Seteo de estilo de los algoritmos
%\algsetup{indent=2em}

\newcommand{\imagen}[3]
{
	\begin{figure}[htbp]
	  \centering
	    \includegraphics[scale=0.5]{#1}
	  \caption{#3}
	\end{figure}
}

\newcommand{\nat}{\mathds{N}}
\newcommand{\algoritmo}[3]{\noindent {\bf\underline{#1}:} #2 $\longrightarrow$ #3}
\newcommand{\superindice}[1]{$^\textrm{{\tiny #1}}$}
\newcommand{\subsubsubsection}[1]{

{\bf\small #1}

}
\newcommand{\negrita}[1]{{\bf #1}}

\renewcommand\floatpagefraction{.9}
\renewcommand\topfraction{.9}
\renewcommand\bottomfraction{.9}
\renewcommand\textfraction{.1}   
\setcounter{totalnumber}{50}
\setcounter{topnumber}{50}
\setcounter{bottomnumber}{50}

%%%%%%%%%%%%%%%%%%%%%%%%%%%%%%%%%%%%%%%%%%%%%%%%%%%%%%%%%%%%%%
%%%%%%%%%%%%%%%%%%%%%%%%%%%%%%%%%%%%%%%%%%%%%%%%%%%%%%%%%%%%%%%%%%%%%%%%%%%%%%%%%%%%
%%%%%   Inicio del documento
%%%%%%%%%%%%%%%%%%%%%%%%%%%%%%%%%%%%%%%%%%%%%%%%%%%%%%%%%%%%%%%%%%%%%%%%%%%%%%%%%%%%
%%%%%%%%%%%%%%%%%%%%%%%%%%%%%%%%%%%%%%%%%%%%%%%%%%%%%%%%%%%%%%

\begin{document}

%%%%%%%%%%%%%%%%%%%%%%%%%%%%%%%%%%%%%%%%%%%%%%%%%%%%%%%%%%%%%%%%%%%%%%
% Caratula
%%%%%%%%%%%%%%%%%%%%%%%%%%%%%%%%%%%%%%%%%%%%%%%%%%%%%%%%%%%%%%%%%%%%%%
\materia{Sistemas Operativos}
\submateria{Primer Cuatrimestre de 2007}
\titulo{Trabajo Pr'actico (parte I)}
\subtitulo{}
\integrante{Freijo, Diego}{4/05}{giga.freijo@gmail.com}
\integrante{Giusto, Maximiliano}{486/05}{maxi.giusto@gmail.com}
\integrante{Iacobacci, Ignacio}{322/02}{iiacobac@gmail.com}
\maketitle

\subsection*{Palabras Clave}
Minix, Kernell, Scheduler, Memoria, Perifericos, Drivers

%%%%%%%%%%%%%%%%%%%%%%%%%%%%%%%%%%%%%%%%%%%%%%%%%%%%%%%%%%%%%%%%%%%%%%
% Indice
%%%%%%%%%%%%%%%%%%%%%%%%%%%%%%%%%%%%%%%%%%%%%%%%%%%%%%%%%%%%%%%%%%%%%%
\clearpage
\tableofcontents
\newpage

\section{Soluciones}
\subsection{Ejercicio 1}
\subsubsection{Layout Minix}
\imagen{img/LayoutMinix.png}{14}{Los niveles de MINIX}

\subsubsection{Kernel}
\begin{itemize}
\item SYS\_FORK: Informa al kernel que un proceso es forkeado
\item SYS\_NEWMAP: Permite al Memory Manager setear una porci'on de memoria para un proceso
\item SYS\_GETMAP: Permite al Memory Manager tomar una porci'on de memoria de un proceso
\item SYS\_EXEC: Setea el contador de programa y el puntero al stack luego de realizar un EXEC
\item SYS\_XIT: Informa al kernel que un proceso ha terminado
\item SYS\_GETSP: El llamador pide un el puntero al stack de un proceso
\item SYS\_TIMES: El llamador pide contar las veces de un proceso
\item SYS\_ABORT: Si el File System o el Memory Manager no pueden continuar. Aborta Minix
\item SYS\_FRESH: Empieza con una imagen de proceso nueva durante EXEC
\item SYS\_SENDSIG: Envia una se�al a un proceso
\item SYS\_SIGRETURN: Se�alizaci'on estilo POSIX
\item SYS\_KILL: Mata un proceso, se�al enviada via Memory Manager
\item SYS\_ENDSIG: Finaliza luego de una se�al KILL
\item SYS\_COPY: Pide un bloque de datos para ser copiado entre procesos
\item SYS\_VCOPY: Pide una serie de bloques de datos para ser copiados entre procesos
\item SYS\_GBOOT: Copia el par'ametro de booteo al un proceso
\item SYS\_MEM: Retorna el pr'oximo bloque libre de memoria f'isica
\item SYS\_UMAP: Computa la direccion f'isica de una direcci'on virtual.
\item SYS\_TRACE: Retorna una operaci'on de traza
\end{itemize}
	
\subsubsection{Memory Manager}
\begin{itemize}
\item FORK: Crea un nuevo proceso
\item EXIT: Termina un proceso
\item WAIT: Detiene un proceso a la espera de una se�al
\item WAITPID: Detiene un proceso a la espera de una se�al
\item BRK: Cambia el tama�o del segmento data
\item EXEC: Ejecuta un archivo
\item KILL: Envia se�al a proceso, usualmente para matar al proceso
\item ALARM: Programa se�al luego de un tiempo especificado
\item PAUSE: Detiene hasta se�al
\item SIGACTION: Administra las se�ales
\item SIGSUSPEND: Suspende hasta se�al
\item SIGPENDING: Reporta pendientes de se�al
\item SIGMASK: Manipula la m'ascara de las se�ales
\item SIGRETURN: Se ejecuta cuando el MM termina manejo de se�ales. Sirve para restablecer contextos.
\item GETUID: Retorna id del usuario
\item GETGID: Retorna id del grupo
\item GETPID: Retorna id del proceso
\item SETUID: Setea id del usuario
\item SETGID: Setea id del grupo
\item SETSID: Crea una nueva sesi'on de un proceso
\item GETPGRP: Retorna el id del grupo del proceso
\item PTRACE: Traza del proceso
\item REBOOT: Apaga el sistema o reinicia
\item KSIG: Acepta se�al que se origina en el Kernel
\end{itemize}

\subsubsection{File System}
\begin{itemize}
\item ACCESS: Determina accesibilidad hacia un archivo
\item CHDIR: Cambia el directorio del trabajo actual
\item CHMOD: Cambia el modo de un archivo
\item CHOWN: Cambia el due�o de un archivo
\item CHROOT: Cambia el directorio del usuario root
\item CLOSE: Borra un descriptor
\item CREAT: Crea un nuevo archivo
\item DUP: Duplica un descriptor
\item FCNTL: Ejecuta varias funciones relacionadas con el archivo descriptor
\item FSTAT: Retorna estado de un archivo abierto
\item IOCTL: Ejecuta varias funciones relacionadas con archivos especiales de tipo car'acter ( como son las terminales )
\item LINK: Crea un link a un archivo
\item LSEEK: Mueve puntero de lectura/escritura
\item MKDIR: Crea un directorio
\item MKNOD: Crea un nuevo inodo
\item MOUNT: Monta un file system
\item OPEN: Abre un archivo para lectura o escritura o crea un nuevo archivo
\item PIPE: Crea un canal de comunicaci'on entre procesos
\item READ: Lee entrada
\item RENAME: Renombra un archivo
\item RMDIR: Remueve un directorio
\item STAT: Retorna estado de la ruta de un archivo
\item STIME: Setea fecha y hora
\item SYNC: Actualiza buffers sucios y super-bloques
\item TIME: Obtiene fecha y hora
\item TIMES: Retorna informaci'on relacionada con los tiempos de un proceso
\item UMASK: Setea la m'ascara de un archivo de un proceso
\item UMOUNT: Desmonta un file system
\item UNLINK: Remueve una entrada de directorio
\item UTIME: Actualiza informaci'on relacionada con los tiempos de un proceso
\item WRITE: Escribe salida
\item UNPAUSE: Env'ia se�al a proceso para ver si est'a suspendido
\item REVIVE: Marca un proceso suspendido como ejecutable
\end{itemize}


b) Instalamos Minix sobre una imagen .vhd.
'Esta fue creada con el entorno Virtua Pc 2004 de Microsoft. 
La intensi'on era generar una imagen que pueda ser corrida en dicha aplicaci'on dado que en sucesivos 
intentos de instalaci'on, el Minix acusaba un ''File System Panic'' y abortaba la misma.
Se us'o pues para la instalacion Qemu 4 con el Qemu Manager de Interfaz Gr'afica. Sin embargo, esta
versi'on present'o problemas al cargar el 'ultimo disco de instalaci'on y fue necesario utilizar la 
versi'on que presenta la p'agina de la materia para concluir exitosamente la misma. El resto de los
puntos fueron desarrollados �ntegramente sobre el Qemu 4.

P'asos para la instalaci'on
\begin{enumerate}
\item Bajar el Qemu 4 + Qemu Manager, versi'on para pendrives.

\verb�http://www.davereyn.co.uk/qem/qemumanager40.zip�

\item Bajar e instalar el Microsoft Virtual PC 2004 ( versi'on gratuita )

\item Microsoft Virtual PC $\rightarrow$ Archivo $\rightarrow$ Asistente para disco virtual $\rightarrow$ Siguiente $\rightarrow$
Crear un nuevo equipo virtual $\rightarrow$ Un disco duro virtual $\rightarrow$ Setear nombre y ubicaci'on $\rightarrow$
Tama�o Fijo $\rightarrow$ Setear tama�o 80MB $\rightarrow$ Finalizar.

\item Qemu Manager 4 $\rightarrow$ Crear una nueva m'aquina virtual $\rightarrow$ Setear nombre de m'aquina virtual $\rightarrow$ Sistema Operativo: \verb�Linux Distribution�, RAM requerida 4MB $\rightarrow$ Checkear \verb�Usar unidad virtual existente�    $\rightarrow$ Seleccionar la imagen creada en 3 $\rightarrow$ Salvar maquina virtual.

\item Opcional (crear m'aquina virtual en Virtual PC ) $\rightarrow$ Asistente para nuevo equipo virtual $\rightarrow$
Crear un nuevo equipo virtual $\rightarrow$ Setear nombre y ubicaci'on $\rightarrow$ Setear Sistema Operativo:
\verb�Otro� $\rightarrow$ Ajustar RAM: 4MB $\rightarrow$ Utilizar un disco virtual existente  $\rightarrow$ Seleccionar la imagen creada en 3 $\rightarrow$
Finalizar

\item Usar el archivo 144M.dsk proveido por la c'atedra. Configurar m'aquina virtual 
(doble click en m'aquina creada en 4) $\rightarrow$ Configuraci'on de disco $\rightarrow$ Disco Floppy A: (fda), 
selecionar 144M.dsk $\rightarrow$ Checkear: \verb�Bootear desde disco floppy� $\rightarrow$ Lanzar maquina virtual.

\item Ingresar \verb�=� en monitor de Minix.

\item Ingresar usuario root.

\item Modificar el archivo setup: \verb�mined /usr/bin/setup�

Reemplazar l'ineas:
\begin{verbatim}
	root = ${primary}a
	usr  = ${primary}c
\end{verbatim}

Por
\begin{verbatim}
	root = hd1
	usr  = hd2
\end{verbatim}
	
\item Ejecutar Setup $\rightarrow$ seleccionar layout $\rightarrow$ select device, presionar \"q\" (no modifica nada) $\rightarrow$
seleccionar hd0 $\rightarrow$ Seleccionar las particiones de hd1 y hd2 $\rightarrow$ Enter.

\item Finalizada la instalaci'on, apagar la m'aquina virtual $\rightarrow$ 

Deschequear: \verb�Bootear desde disco floppy�


\item con el comando partir, generar las imagenes de \verb�USR.TAZ, SYS.TAZ Y CMD.TAZ�.
En l'inea de comando ejecutar en el directorio donde estan los archivos y el ejecutable:
partir [u] usr.taz
partir [s] sys.taz
partir [c] cmd.taz
Ubicar los archivos generados en carpeta media del Qemu

\item Reiniciar m'aquina virtual. 

\item Con \verb0ctrl+alt+20 se abre consola de Qemu. change fda u1.img.
Con \verb0ctrl+alt+10 se vuelve a Minix.

\item Ejecutar \verb0setup /usr0. Repetir 14 para cada vez que se pida el siguiente disco.

\item Repertir 14 y 15 para s1.img y con c1.img

Nota: La instalaci'on de archivo cmd.taz nunca se pudo completar con la versi'on de Qemu usada. 
Se intent'o varias veces con resultados adversos. Se utiliz'o el Qemu proveido por la c'atedra 
para la instalaci'on de este archivo.

\end{enumerate}


\subsection{Ejercicio 2: Herramientas}

Indique que hace el comando make y mknode. C'omo se utilizan 'estos comandos en la instalaci'on de MINIX y en la creaci'on de un nuevo kernel. Para el caso del make muestre un archivo ejemplo y explique que realiza cada uno de los comandos internos del archivo ejemplo.

El comando make sirve para compilar archivos. Make es un programa que es normalmente usado para desarrollar programas grandes que consisten en muchos archivos. Lleva cuenta de cuales archivos objeto dependen de cuales archivos fuentes y de cabecera. Cuando es ejecutado, hace la m'inima cantidad de recompilaciones para obtener el archivo destino actualizado.
El comando make se utiliza en la creaci'on de un nuevo kernel. Luego de modificar alg'un archivo fuente del kernel, uno debe reconstruir el kernel. En el directorio \textbf{/usr/src/tools} est'a el archivo Makefile, necesario para reconstruir el kernel.

El comando mknod sirve para crear archivos especiales ya sea archivos asociados a dispositivos de entrada/salida o bien directorios. El uso de 'este comando est'a reservado a los superusuarios.

A pesar de que en la instalaci'on de MINIX nosotros no tuvimos que, expl'icitamente, usar estos comandos, el comando mknod se utiliz'o para crear los nodos device en el nodo ra'iz: \textbf{/dev/tty}; \textbf{/dev/tty[0-2]}; \textbf{/dev/hd[0-9]}.

\subsection{Ejercicio 3: Comandos b'asicos MINIX/UNIX}
\subsubsection{Pongale password root a root.}

Utilizando el comando passwd:

\begin{verbatim}
    # passwd
    Changing the shadow password of root
    New password:
    Retype password:
    # exit
    Minix Release 2.0 Version 0
    noname login: root
    Password:
    #
\end{verbatim}

\subsubsection{pwd}

Indique qu� directorio pasa a ser su $current\ directory$ si ejecuta:

\subsubsubsection{1.3.2.1. $\sharp$ cd /usr/src}

\begin{verbatim}
    # cd /usr/src
    # pwd
    /usr/src
\end{verbatim}

Utilizando el comando \textbf{pwd} podemos ver en que directorio nos encontramos. En 'este caso el $current\ directory$ es \textbf{/usr/src}.

\subsubsubsection{1.3.2.2. $\sharp$ cd}

\begin{verbatim}
    # cd
    # pwd
    /
    #
\end{verbatim}

Nuevamente utilizamos el comando \textbf{pwd} y el directorio actual es \textbf{/}.

\subsubsubsection{1.3.2.3. �C'omo explica el punto 1.3.2.2.?}

Ejecutar el comando \textbf{cd} sin argumentos cambia el directorio actual al directorio home del usuario. Para el usuario root 'este directorio es el directorio ra'iz del sistema.

\subsubsection{cat}

Cu'al es el contenido del archivo \textbf{/usr/src/.profile} y para qu'e sirve.

\begin{verbatim}
    # cat /usr/src/.profile
    # Login shell profile.
    
    # Environment.
    umask 022
    PATH=/usr/local/bin:/bin:/usr/bin
    PS1="! "
    export PATH
    
    # Erase character, erase line, and interrupt keys.
    stty erase '^H' kill '^U' intr '^?'
    # Check terminal type.
    case $TERM in
    dialup|unknown|network)
            echo -n "Terminal type? ($TERM) "; read term
            TERM="${term:-$TERM}"
    esac
    
    # Shell configuration.
    case "$0" in *ash) . $HOME/.ashrc;; esac
    #
\end{verbatim}

Cada usuario tendr'a 'este script en su home, y cada vez que se "loguee" se ejecutar'a.

\subsubsection{find}

En qu'e directorio se encuentra el archivo \textbf{proc.c}

\begin{verbatim}
    # find / -name proc.c
    /usr/src/kernel/proc.c
\end{verbatim}

\subsubsection{mkdir}

Genere un directorio \textbf{/usr/$<$nombregrupo$>$}

\begin{verbatim}
    #mkdir /usr/grupo8
    # cd /usr
    # ls
    adm bin grupo8  lib   man  preserve src
    ast etc include local mdec spool    tmp
\end{verbatim}

\subsubsection{cp}

Copie el archivo \textbf{/etc/passwd} al directorio \textbf{/usr/$<$nombregrupo$>$}

\begin{verbatim}
    # cp /etc/passwd /usr/grupo8/passwd
    # cd /usr/grupo8
    # ls
    passwd
    #
\end{verbatim}

\subsubsection{chgrp}

Cambie el grupo del archivo \textbf{/usr/$<$grupo$>$/passwd} para que sea other.
\begin{verbatim}
    # pwd
    /usr/grupo8
    # ls -l
    total 1
    -rw-r--r-- 1 root operator 285 Jun 17 22:05 passwd
    # chgrp other passwd
    # ls -l
    total 1
    -rw-r--r-- 1 root other 285 Jun 17 22:05 passwd
    #
\end{verbatim}

\subsubsection{chown}

Cambie el propietario del archivo \textbf{/usr/$<$grupo$>$/passwd} para que sea \textbf{ast}.

\begin{verbatim}
    # chown ast passwd
    # ls -l
    total 1
    -rw-r--r-- 1 ast other 285 Jun 17 22:06 passwd
    #
\end{verbatim}

\subsubsection{chmod}

Cambie los permisos del archivo \textbf{/usr/$<$grupo$>$/passwd} para que:

\begin{itemize}
\item el propietario tenga permisos de lectura, escritura y ejecuci'on
\item el grupo tenga solo permisos de lectura y ejecuci'on
\item el resto tenga solo permisos de ejecuci'on
\end{itemize}

\begin{verbatim}
    # chmod 751 passwd
    # ls -l
    total 1
    -rwxr-x--x 1 ast other 285 Jun 17 22:11 passwd
    #
\end{verbatim}

\subsubsection{grep}

Muestre las lineas que tiene el texto \textbf{include} en el archivo \textbf{/usr/src/kernel/main.c}

\begin{verbatim}
    # grep include /usr/src/kernel/main.c
    #include "kernel.h"
    #include <signal.h>
    #include <unistd.h>
    #include <minix/callnr.h>
    #include <minix/com.h>
    #include "proc.h"
    #
\end{verbatim}

Muestre las lineas que tiene el texto \textbf{POSIX} que se encuentren en todos los archivos \textbf{/usr/src/kernel/}

\begin{verbatim}
# grep POSIX /usr/src/kernel/*
/usr/src/kernel/kernel.h:#define _POSIX_SOURCE 1 /* tell headers to inclu
de POSIX stuff */
/usr/src/kernel/rs232.c: if ((tp->tty_termios.c_lflag & IXON) && rs->oxoff !=
_POSIX_VDISABLE)
/usr/src/kernel/system.c: * SYS_SENDSIG send a signal to a process (POSIX style)
/usr/src/kernel/system.c: * SYS_SIGRETURN complete POSIX-style signalling
/usr/src/kernel/system.c:/* Handle sys_sendsig, POSIX-style signal */
/usr/src/kernel/system.c:/* POSIX style signals require sys_sigreturn to put
things in order before the
/usr/src/kernel/tty.c:/* These Posix functions are allowed to fail if _POSIX_JOB
_CONTROL is
/usr/src/kernel/tty.c: /* _POSIX_VDISABLE is a normal character value, so better
escape it. */
/usr/src/kernel/tty.c: if (ch == _POSIX_VDISABLE) ch |= IN_ESC;
#
\end{verbatim}

\subsubsection{su}

\subsubsubsection{1.3.11.1. Para qu'e sirve?}

Permite convertir un usuario en otro sin tener que desconectarse del sistema.
Por defecto lo convierte en superuser. Le pedir'a la password correspondiente.

\subsubsubsection{1.3.11.2. Qu'e sucede si ejecuta el comando \textbf{su} estando logueado como \textbf{root}?}

Se genera un shell nuevo. Como estabamos logueados como root, no pidi'o password.

\subsubsubsection{1.3.11.3. Genere una cuenta de $<$usuario$>$}
%el usuario es Maxi pero me pareci� mejor poner pepe

\begin{verbatim}
    # adduser pepe other /usr/pepe
    cpdir /usr/ast /usr/pepe
    chown -R 10:3 /usr/pepe
    echo pepe::0:0::: >>/etc/shadow
    echo pepe:##pepe:10:3:pepe:/usr/pepe: >>/etc/passwd
    The new user pepe has been added to the system. Note that the password,
    full name, and shell may be changed with the commands passwd(1), chfn(1),
    and chsh(1). The password is now empty, so only console logins are possible.
\end{verbatim}

\subsubsubsection{1.3.11.4. Entre a la cuenta $<$usuario$>$ generada}

\begin{verbatim}
    Minix Release 2.0 Version 0
    noname login: pepe
    $
\end{verbatim}

No pidi'o Password pues el usuario pepe no tiene a'un una seteada.
 
\subsubsubsection{1.3.11.5. Repita los comandos de 1.3.11.2}

Estando logueado como pepe ejecutamos el comando \textbf{su}.

\begin{verbatim}
    $ su
    Password:
    #
\end{verbatim}

Nos pide la password de root pues el usuario pepe es parte del grupo other. Si fuera parte del grupo operator no la pedir'ia.

\subsubsection{passwd}

\subsubsubsection{1.3.12.1. Cambie la password del usuario nobody}

\begin{verbatim}
    # passwd nobody
    Changing the password of nobody
    New password:
    Retype password:
    #
\end{verbatim}

\subsubsubsection{1.3.12.2. Presione las teclas ALT-F2 y ver'a otra sesi'on MINIX. Loguearse como nobody }

\begin{verbatim}
    Minix Release 2.0 Version 0
    noname login: nobody
    Password:
    $
\end{verbatim}

Nos logueamos con password nobody.

\subsubsubsection{1.3.12.3. Ejecutar el comando su}

\begin{verbatim}
    $ su
    Password:
    #
\end{verbatim}

\subsubsubsection{1.3.12.3.1. Qu'e le solicita?}

Solicita la password de root.

\subsubsubsection{1.3.12.3.2. Sucede lo mismo que en 1.3.11.2?�Por qu'e?}

No, cuando se realiz'o lo mismo con el usuario \textbf{root} no fue necesario ingresar la password. Esto se debe a que el usuario \textbf{nobody} pertenece al grupo \textbf{nogroup}, quienes no tienen permisos para loguearse como \textbf{root} y, de hecho, casi no poseen ningun tipo de privilegios.

\subsubsection{rm}

Suprima el archivo \textbf{/usr/$<$grupo$>$/passwd}.
\begin{verbatim}
    # ls /usr/grupo8
    passwd
    # rm /usr/grupo8/passwd
    # ls /usr/grupo8
    #
\end{verbatim}

\subsubsection{ln}

Enlazar el archivo \textbf{/etc/passwd} a los siguientes archivos \textbf{/tmp/contra1} \textbf{/tmp/contra2}.
Hacer un ls -l para ver cuantos enlaces tiene \textbf{/etc/passwd}.

\begin{verbatim}
    # ls -l /etc/passwd
    -rw-r--r-- 1 root operator 314 Apr 15 2007 /etc/passwd
    # ln /etc/passwd /tmp/contra1
    # ln /etc/passwd /tmp/contra2
    # ls -l /etc/passwd
    -rw-r--r-- 3 root operator 314 Apr 15 2007 /etc/passwd
    #
\end{verbatim}

En el listado del archivo, la segunda columna (luego de los permisos) indica la cantidad de links que tiene ese archivo.

\subsubsection{mkfs}

Genere un Filesystem MINIX en un diskette

\begin{verbatim}
    # mkfs /dev/fd0
\end{verbatim}

\subsubsection{mount}

\subsubsubsection{M'ontelo en el directorio \textbf{/mnt}.}

\begin{verbatim}
    # mount /dev/fd0 /mnt
    /dev/fd0 is read-write mounted on /mnt
    #
\end{verbatim}

\subsubsubsection{Presente los filesystems que tiene montados.}

\begin{verbatim}
    # mount
    /dev/hd1 is root device
    /dev/hd2 is mounted on /usr
    /dev/fd0 is mounted on /mnt
    #
\end{verbatim}


\subsubsection{df}

\subsubsubsection{Qu'e espacio libre y ocupado tienen todos los filesystems montados?(en KBYTES)}

\begin{verbatim}
# df
Device    Inodes Inodes Inodes  Blocks Blocks Blocks  Mounted   V Pr
           total  used   free    total  used   free      on         
           -----  -----  -----   -----  -----  -----    ------   - --
/dev/hd1    496    212    284     1480    700    780    /        2 rw
/dev/hd2  12528   3309   9219    75096  27443   47653   /usr     2 rw
/dev/fd0    480      1    479     1440     35    1405   /mnt     2 rw
#
\end{verbatim}

En todos los dispositivos, el tama'no de bloque es 1KB, por lo tanto el espacio disponible en KB es el indicado por la columna \textbf{Blocks free} y el espacio ocupado es el indicado por la columna \textbf{Blocks used}.

\subsubsection{ps}

\subsubsubsection{1.3.18.1. �Cu�ntos procesos de usuario tiene ejecutando?}

\begin{verbatim}
    # ps -a
    PID TTY  TIME CMD
     41  co  0:00 -sh
     42  c1  0:00 getty
     47  c0  0:00 ash
     48  c0  0:00 ps -a
\end{verbatim}

Los procesos de usuario que se est'an ejecutando son 4 (incluyendo al \textbf{ps}).

\subsubsubsection{1.3.18.2. Indique cu'antos son del sistema}

\begin{verbatim}
    # ps -ax
    PID TTY  TIME CMD
      0   ?  0:00 TTY
      0   ?  0:00 SCSI
      0   ?  0:00 WINCH
      0   ?  0:00 SYN_AL
      0   ?  8:35 IDLE
      0   ?  0:00 PRINTER
      0   ?  0:00 FLOPPY
      0   ?  0:00 MEMORY
      0   ?  0:00 CLOCK
      0   ?  0:01 SYS
      0   ?  0:06 HARDWAR
      0   ?  0:00 MM
      0   ?  0:04 FS
      1   ?  0:00 INIT
     40  co  0:00 -sh
     27   ?  0:00 update
     41  c1  0:00 getty
     47  co  0:00 ash
     57  co  0:00 ps -ax
    #
\end{verbatim}

Aqu'i se listan todos los procesos incluyendo tambi'en a los de usuario. En total hay 18 procesos ejecut'andose, 4 son de usuario y 15 del sistema.

\subsubsection{umount}

\subsubsubsection{1.3.19.1. Desmonte el Filesystem del directorio /mnt}

\begin{verbatim}
    # umount /dev/fd0
    /dev/fd0 unmounted from /mnt
    #
\end{verbatim}

\subsubsubsection{1.3.19.2. Monte el Filesystem del diskette como read-only en el directorio /mnt}

\begin{verbatim}
    # mount /dev/fd0 /mnt -r
    /dev/fd0 is read-only mounted on /mnt
    #
\end{verbatim}

\subsubsubsection{1.3.19.3. Desmonte el Filesystem del directorio /mnt}

\begin{verbatim}
    # umount /dev/fd0
    /dev/fd0 unmounted from /mnt
    #
\end{verbatim}

\subsubsection{fsck}

\subsubsubsection{Chequee la consistencia de Filesystem del diskette}

\begin{verbatim}
    # fsck /dev/fd0

    Checking zone map
    Checking inode map
    Checking inode list
    
    blocksize =  1024        zonesize  =  1024
    
         0    Regular files
         1    Directory
         0    Block special files
         0    Character special files
       479    Free inodes
         0    Named pipes
         0    Symbolic links
      1405    Free zones
    #
\end{verbatim}

\subsubsection{dosdir}

\subsubsubsection{Tome un diskette formateado en DOS con archivos y ejecute $dosdir a$}

\begin{verbatim}
    #dosdir a
    dosdir: cannot open /dev/dosA: no such file or directory
    #
\end{verbatim}

\subsubsubsection{Ejecute los comandos necesarios para que funcione correctamente el comando anterior}
El diskette con formato DOS est'a en \textbf{/dev/fd1}

\begin{verbatim}
    #dosdir fd1
    HOLADOS.TXT
    ABMUSR
    GETNEXTG
    MACHAQUE
    #
\end{verbatim}

\subsubsection{dosread}

\subsubsubsection{Copie un archivo de texto desde un diskette DOS al directorio \textbf{/tmp}}

\begin{verbatim}
    #dosread fd1 HOLADOS.TXT
    hola mundo!

    #dosread -a fd1 HOLADOS.TXT > /tmp/holaminix.txt
    #cat /tmp/holaminix.txt
    hola mundo!

    #
\end{verbatim}

\subsubsection{doswrite}

\subsubsubsection{Copie el archivo \textbf{/etc/passwd} al diskette DOS}

\begin{verbatim}
    #doswrite -a fd1 passwd < /etc/passwd
    #dosdir fd1
    HOLADOS.TXT
    ABMUSR
    GETNEXTG
    MACHAQUE
    PASSWD
    #
\end{verbatim}

\subsection{Ejercicio 4: Uso de STDIN, STDOUT, STDERR y PIPES}
\subsubsection{STDOUT}
\subsubsubsection{1.4.1.1. Conserve el archivo /usr/$<$grupo$>$/fuentes.txt la salida del comando ls que muestra todos los archivos del directorio /usr/src y de los subdirectorios bajo /usr/src}

\begin{verbatim}
    # cd /usr/grupo8
    # ls -al /usr/src > /usr/grupo8/fuentes.txt
    #
\end{verbatim}

\subsubsubsection{1.4.1.2. Presente cuantas lineas, palabras y caracteres tiene /usr/$<$grupo$>$/fuentes.txt}

\begin{verbatim}
    # wc /usr/grupo8/fuentes.txt
         25    218   1455 /usr/grupo8/fuentes.txt
    #
\end{verbatim}

El archivo \textbf{/usr/grupo8/fuentes.txt} tiene 25 l�neas, 218 palabras y 1455 caracteres.

\subsubsection{STDOUT}
\subsubsubsection{1.4.2.1. Agregue el contenido, ordenado alfab'eticamente, del archivo /etc/passwd al final del archivo /usr/$<$grupo$>$/fuentes.txt}

\begin{verbatim}
    # sort /etc/passwd >> /usr/grupo8/fuentes.txt
\end{verbatim}

\subsubsubsection{1.4.2.2. Presente cu'antas lineas, palabras y caracteres tiene usr/$<$grupo$>$/fuentes.txt}

\begin{verbatim}
    # wc /usr/grupo8/fuentes.txt
         34    234   1769 /usr/grupo8/fuentes.txt
    #
\end{verbatim}

El archivo \textbf{/usr/grupo8/fuentes.txt} tiene 34 l�neas, 234 palabras y 1769 caracteres.

\subsubsection{STDIN}

\subsubsubsection{1.4.3.1. Genere un archivo llamado /usr/$<$grupo$>$/hora.txt usando el comando echo con el siguiente contenido: 2355}

\begin{verbatim}
    # echo 2355 > /usr/grupo8/hora.txt
\end{verbatim}

\subsubsubsection{1.4.3.2. Cambie la hora del sistema usando el archivo /usr/$<$grupo$>$/hora.txt generado en 1.4.3.1}

\begin{verbatim}
    # date -q < /usr/grupo8/hora.txt
\end{verbatim}

\subsubsubsection{1.4.3.3. Presente la fecha del sistema}

\begin{verbatim}
    # date
    Mon Aug 20 23:55:07 MET DST 2007
    #
\end{verbatim}

\subsubsection{STDERR}

\subsubsubsection{Guarde el resultado de ejecutar el comando dosdir k en el archivo /usr/$<$grupo$>$/error.txt. Muestre el contenido de /usr/<grupo>/error.txt.}

\begin{verbatim}
    # dosdir k > /usr/grupo8/error.txt 2>&1
    # cat /usr/grupo8/error.txt
    dosdir: cannot open /dev/dosK: No such file or directory
    #
\end{verbatim}

\subsubsection{PIPES}
Posici'onese en el directorio \textbf{/}(directorio ra'iz), una vez que haya hecho eso:
\subsubsubsection{1.4.5.1. Liste en forma amplia los archivos del directorio /usr/bin que comiencen con la letra s. Del resultado obtenido, seleccione las l'ineas que contienen el texto sync e informela cantidad de caracteres, palabras y l'ineas.}
Nota 1: Est'a prohibido, en este 'item, usar archivos temporales de trabajo. 

Nota 2: Si le da error, es por falta de memoria, cierre el proceso de la otra sesi'on, haga un kill sobre los procesos update y getty.

\begin{verbatim}
    # pwd
    /
    # ls -lc /usr/bin/s* | grep sync | wc
          2     18    140
    #
\end{verbatim}

2 l�neas, 18 palabras y 140 caracteres.

\subsection{Ejercicio 5}
El comando que realiza las tareas es \textbf{abmusr} y su uso es

Para agregar al usuario arturo: \verb0abmusr -a arturo grupoarturo /usr/arturo0

Para borrar al usuario arturo: \verb0abmusr -b arturo0

\subsection{Ejercicio 6: Ejecuci'on de procesos en Background}
Crear el siguiente programa 
\begin{verbatim}
/usr/src/loop.c
#include <stdio.h>
int main()
{
  int i, c;
  while(1)
  {
    c = 48 + i;
    printf("%d",c);
    i++;
    i = i % idgrupo;
  }
}
\end{verbatim}

\subsubsubsection{Compilarlo. El programa compilado debe llamarse loop. Indicando a la macro idgrupo el valor de su grupo.}

Creamos el programa y lo guardamos en el directorio \textbf{/usr/src}. Para compilar el programa realizamos los siguientes pasos:

\begin{verbatim}
    # pwd
    /usr/src
    # cc loop.c -oloop
    #
\end{verbatim}

\subsubsubsection{(a) Correrlo en foreground �Qu'e sucede? Mate el proceso con el comando kill.}

Ejecutamos el programa en foreground y notamos que la pantalla se llena de n'umeros, es decir, se produce una impresi'on indefinida de n'umeros en pantalla. La consola queda inutilizable ya que el programa est'a en un loop infinito.

\begin{verbatim}
# ./loop
2535455484950515253545548495051525354554849505152535455484950515253545
2535455484950515253545548495051525354554849505152535455484950515253545
2535455484950515253545548495051525354554849505152535455484950515253545
2535455484950515253545548495051525354554849505152535455484950515253545
2535455484950515253545548495051525354554849505152535455484950515253545
2535455484950515253545548495051525354554849505152535455484950515253545
\end{verbatim}

Como necesitamos saber el PID del proceso para usar kill, creamos una nueva consola mediante Alt+F2 y en ella ejecutamos los siguientes comandos:

\begin{verbatim}
    Minix Release 2.0 Version 0
    noname login: root
    Password:
    # ps 
    PID TTY  TIME CMD
     40  co  0:00 -sh
     41  c1  0:00 -sh
     47  co  0:00 ash
     49  co  0:46 ./loop
     53  c1  0:00 ash
     54  c1  0:00 ps
    # kill 49
    #
\end{verbatim}

Esto causa que en la primer consola el programa finalice con la leyenda \textbf{Terminated}.

\begin{verbatim}
2484950515248495051524849505152484950515248495051524849505152484950515
2484950515248495051524849505152484950515248495051524849505152484950515
2484950515248495051524849505152484950515248495051524849505152484950515
2484950515248495051524849505152484950515248495051524849505152484950515
24849505152484950515248Terminated
\end{verbatim}

\subsubsubsection{(b) Ahora ejec'utelo en background}
\indent \textbf{/usr/src/loop $>$ /dev/null \&}
\subsubsubsection{�Qu'e se muestra en la pantalla?}

Al ejecutarlo en background, el shell devuelve el ID del proceso. Al estar siendo redirigida la salida a null no imprime nada en pantalla.

\begin{verbatim}
# pwd
/usr/src
# ./loop > /dev/null &
#
\end{verbatim}

Cabe aclarar que si el usuario logueado cambi'o su shell, por ejemplo cambi'o a \textbf{ash}, no visualizar'a el ID del proceso y deber'a realizar un \textbf{ps -a} para obtenerlo.

\subsubsubsection{�Qu'e sucede si presiona la tecla F1? �Qu� significan esos datos?}

La tecla F1, muestra una tabla con los procesos del sistema. 'Esta tabla contiene datos del proceso y datos del sistema. Presenta las siguientes columnas:

\begin{itemize}

\item \textbf{pid}: El identificador del proceso. Puede ser el PID asignado por el MM, o puede ser el spot en la tabla de procesos, es decir el p nr, si este es menor a 0.
\item \textbf{pc}: Indica el valor del Program Counter en el momento en que el proceso fue bloqueado.
\item \textbf{sp}: Indica la direcci'on del puntero al tope del stack del proceso.
\item \textbf{flag}: Indica el valor del word de los flags. Si no hay ning'un flag activado, es decir, si este valor es 0, el proceso puede ser ejecutado, en caso contrario se encuentra bloqueado o a'un no fue inicializado.
\item \textbf{user}: Tiempo que el proceso estuvo utilizando el procesador. Est'a medido en "ticks", aproximadamente hay 60 ticks por segundo (59,7 medido, pero bajo Bochs por lo que es inexacto).
\item \textbf{sys}: Tiempo de ejecuci'on de rutinas de sistema relacionadas a la administraci'on de 'este proceso. Tambi'en medido en "ticks"
\item \textbf{text}: Direcci'on f'isica donde comienza el segmento TEXT, o de c'odigo, asignado al proceso.
\item \textbf{data}: Direcci'on f'isica donde comienza el segmento DATA, o de datos, asignado al proceso.
\item \textbf{size}: Tama'no del espacio ocupado en memoria por el proceso en KB.
\item \textbf{recv}: Si el proceso se encuentra bloqueado en espera de el envio o recepci'on de un mensaje, este campo indica quien es el receptor o emisor de dicho mensaje. Caso contrario, est'a en blanco.
\item \textbf{command}: Indica el nombre del proceso.
\end{itemize}

\subsubsubsection{�Qu'e sucede si presiona la tecla F2? �Qu'e significan esos datos?}

Al presionar F2 se obtiene informaci'on sobre el mapa de memoria de los procesos
(estructura mem map en \textbf{/usr/include/minix/type.h}). La tabla presenta
las siguientes columnas:

\begin{itemize}
\item \textbf{PROC}: Indica el slot en la tabla de procesos, es decir el p nr.
\item \textbf{NAME}: Indica el nombre del proceso.
\item \textbf{TEXT}: Para el segmento de c'odigo indica la direcci'on virtual, la direcci'on f'isica y el tama'no, en ese orden.
\item \textbf{DATA}: Para el segmento de datos indica la direcci'on virtual, la direcci'on f'isica y el tama'no, en ese orden.
\item \textbf{STACK}: Para el segmento de stack, o pila, indica la direcci'on virtual, la direcci'on f'isica y el tama'no, en ese orden.
\item \textbf{SIZE}: Tama'no del espacio ocupado en memoria por el proceso.
\end{itemize}

\subsection{Ejercicio 7}
\subsubsection{Inciso a}
Ejecutar

\verb�# make install�
    
Se compilan todo los archivos del kernel, mm o fs que no tengan su archivo .o generado.

\begin{verbatim}
cd ../kernel && exec make - install
cd keymaps && make - install
make: 'install' is up to date
cd ../mm && exec make - install
make: 'install' is up to date
cd ../fs && exec make - install
make: 'install' is up to date
cd ../inet && exec make - install
make: 'install' is up to date
\end{verbatim}

Ejecutar

\verb�# make hdboot�

Para generar una nueva imagen de Minix

\begin{verbatim}
cd ../kernel && exec make -
make: 'kernel' is up to date
cd ../mm && exec make -
exec cc -c -I/usr/include main.c
exec cc -c -I/usr/include forkexit.c
exec cc -c -I/usr/include break.c
exec cc -c -I/usr/include exec.c
exec cc -c -I/usr/include signal.c
exec cc -c -I/usr/include alloc.c
exec cc -c -I/usr/include utility.c
exec cc -c -I/usr/include table.c
exec cc -c -I/usr/include putk.c
exec cc -c -I/usr/include trace.c
exec cc -c -I/usr/include getset.c
exec cc -c -I/usr/include newcall.c
exec cc -c -I/usr/include llamsistema.c
exec cc -c -I/usr/include semaf.c
exec cc -o mm -i main.o forkexit.o break.o exec.o \
	signal.o alloc.o utility.o table.o \
	putk.o trace.o getset.o newcall.o \
	llamsistema.o semaf.o
install -S 256w mm
cd ../fs && exec make -
exec cc -c -I/usr/include main.c
exec cc -c -I/usr/include open.c
exec cc -c -I/usr/include read.c
exec cc -c -I/usr/include write.c
exec cc -c -I/usr/include pipe.c
exec cc -c -I/usr/include device.c
exec cc -c -I/usr/include path.c
exec cc -c -I/usr/include mount.c
exec cc -c -I/usr/include link.c
exec cc -c -I/usr/include super.c
exec cc -c -I/usr/include inode.c
exec cc -c -I/usr/include cache.c
exec cc -c -I/usr/include cache2.c
exec cc -c -I/usr/include filedes.c
exec cc -c -I/usr/include stadir.c
exec cc -c -I/usr/include protect.c
exec cc -c -I/usr/include time.c
exec cc -c -I/usr/include misc.c
exec cc -c -I/usr/include utility.c
exec cc -c -I/usr/include table.c
exec cc -c -I/usr/include putk.c
exec cc -c -I/usr/include othercall.c
exec cc -o fs -i main.o open.o read.o write.o pipe.o \
	device.o path.o mount.o link.o super.o inode.o \
	cache.o cache2.o filedes.o stadir.o protect.o time.o \
	lock.c misc.o utility.o table.o putk.o othercall.o
install -S 512w fs
installboot -image image ../kernel/kernel ../mm/mm ../fs/fs  init
    text    data     bss     size
   53248    8964   40076   102288  ../kernel/kernel
   14096    1244   30736    46076  ../mm/mm
   29184    2228  108084   139496  ../fs/fs
    6828    2032    1356    10216  init
  ------  ------  ------  -------
  103356   14468  180252   298076  total
exec sh mkboot hdboot
rm /dev/hd1:/minix/2.0.2SISTEMASO
cp image /dev/hd1:/minix/2.0.2SISTEMASOPERATIVOS-UBA-FCENr1
Done.
\end{verbatim}


Ejecutar

\verb�# reboot�

 para reiniciar con la nueva version de kernel

\subsubsection{Inciso a}
Se crea directorio \verb�/minix2� y se copia desde \verb�/minix� el archivo 2.0.0. Al iniciar minix presionar ESC. 
Escribir 

\verb�> image=minix�

Si se quiere iniciar con la nueva versi�n del kernel
	
Escribir 	

\verb�>image= minix2�

Si se quiere iniciar con la versi�n original.

\verb�>boot�

Para iniciar


\subsection{Ejercicio 8}
En MINIX tenemos las siguientes caracter'isticas:
\begin{itemize}
\item Administraci'on de la memoria: se maneja por pol'itica de memoria segmentada variable con primera zona. Es decir, se asigna el primer lugar donde exista el espacio necesario. 
\item Administraci'on del procesador: la administraci'on de MINIX es una triple cola de prioridad (es decir, es una multicola) donde se utiliza FIFO para la del Kernell (de mayor prioridad), FIFO para la de System Tasks como el MM y FS (de prioridad menor) y Round Robin para las colas del usuario (prioridad 'ultima).
\item Administraci'on de E/S: en la 1er capa (la de drivers) hay un proceso (driver) por cada dispositivo en el sistema. Cuando un proceso de usuario quiere acceder a un dispositivo, 'este lo realiza a travez de la 2da capa por alguno de los servicios alli expuestos (MM, FS o red) y 'estos se comunican con los drivers a travez de mensajes (los cuales se comunican con el kernell en la capa 0 por otros mensajes).
\item Administraci'on de FS: el sistema de archivos de MINIX posee seis componentes:

\begin{itemize}
\item El bloque de booteo, que esta siempre almacenado en el primer bloque. Contiene la informaci'on sobre como iniciar el sistema al encenderse.
\item El segundo bloque es el Superbloque y almacena informacion sobre el FS que permite al SO localizar y entender otras estructuras de sistemas de archivos (n'umero de inodos y zonas, el tama'no de dos bitmaps y el bloque de inicio del 'area de informaci'on).
\item El bitmap de inodo es un simple mapa de inodos que localiza cuales est'an en uno y cuales est'an libres represent'andolos con un bit. 
\item El bitmap de zona trabaja de la misma forma que el bitmap de inodo, excepto que localiza las zonas.
\item Los inodos de 'area. Cada archivo o directorio est'a representado como un inodo, el cual almacena metadata incluyendo tipo (archivo, directorio, bloque, char, pipe), ids de usuario y grupo, tres timestamps que graban la fecha y hora de 'ultimo acceso, 'ultima modificaci'on y 'ultimo cambio de estado. Un inodo adem'as contiene una lista de direcciones que apuntan a las zonas en el 'area de informaci'on donde el archivo o directorio es'ta ubicado.
\item El 'area de datos es el componente mas grande del sistema de archivos, usando la mayor'ia del espacio. 
\end{itemize}

\end{itemize}


Y en LINUX tenemos lo siguiente:
\begin{itemize}
\item Administraci'on de la memoria: utiliza paginaci'on por demanda, por lo que brinda:

\begin{itemize}
\item Grandes espacios de direccionamiento
El sistema operativo hace aparentar al sistema como si tuviese mas memoria que la que realmente tiene. La memoria virtual puede ser varias veces mas grandes que la memoria fisica en el sistema.

\item Protecci'on
Cada proceso en el sistema tiene su propio espacio de direccionamiento. 'Estos est'an completamente separados entre s'i y por eso un proceso corriendo una aplicacion no puede afectar otro. Adem'as, los mecanismos de  hardware de memoria virtual permite que 'areas de memoria sean protegidas contra escritura para proteger el c'odigo y los datos de ser sobre escritos por otras aplicaciones. 

\item Mapeo de memoria
El mapeo de memoria es usado para mapear archivos al espacio de direccionamiento de un proceso. Es decir, los contenidos de un archivo son enlazados directamente en el espacio de direccionamiento virtual del proceso.

\item Asignaci'on justa de la memoria f'isica
El subsistema de administraci'on de memoria le permite obtener a cada proceso en ejecuci'on en el sistema una parte justa de la memoria f'isica.

\item Memoria virtual compartida
Although virtual memory allows processes to have separate (virtual) address spaces, there are times when you need processes to share memory. For example there could be several processes in the system running the bash command shell. Rather than have several copies of bash, one in each processes virtual address space, it is better to have only one copy in physical memory and all of the processes running bash share it. Dynamic libraries are another common example of executing code shared between several processes. 
Shared memory can also be used as an Inter Process Communication (IPC) mechanism, with two or more processes exchanging information via memory common to all of them. Linux supports the Unix TM System V shared memory IPC. 

A pesar que memoria virtual permite a los procesos tener separados espacios (virtuales) de direccionamiento, hay veces que necesitan compartir memoria (por ejemplo, librer'ias din'amicas). Por lo que, en lugar de duplicar datos, se utiliza un mecanismo de comunicaci'on entre procesos (IPC) para que dos o m'as procesos intercambien informaci'on a trav'ez de memoria en com'un. El soporte en Linux se llama TM System V shared memory IPC. 
\end{itemize}

\item Administraci'on del procesador: se basa en una pol'itica Round-Robin (en su nomenclatura, \emph{time-sharing technique}) donde para cada proceso existe una cuota de tiempo (quantum, o tambi'en llamado \emph{timeslice}) que puede utilizar del procesador por 'epoca. Al comienzo de cada 'epoca, el timeslice es recomputado por proceso y cada uno de ellos podr'a utilizar el procesador 'este tiempo asignado a cada uno durante la 'epoca. Notar que si un proceso utiliza demasiadas E/S y por ende no consume todo su timeslice al final de la 'epoca, 'este sobrante es agregado en la pr'oxima 'epoca.
%%%%%%%%%5

\item Administraci'on de E/S: 
Linux soporta tres tipos de dispositivo de hardware: caracter, bloque y red. Los dispositivos en modo caracter se leen y se escriben directamente sin buffering, por ejemplo los puertos seriales /dev/cua0 y /dev/cua1 del sistema. Los dispositivos en modo bloque pueden ser solamente escritos le'idos en en los m'ultiplos del tama'no de bloque, de t'ipicamente 512 o 1024 bytes. Los dispositivos en modo bloque son accedidos por el buffer y sede forma aleatoria, es decir, cualquier bloque puede ser le'ido o ser escrito no importa d'onde est'a en el dispositivo. Los dispositivos en modo bloque se pueden acceder v'ia un archivo especial del dispositivo pero se suele utilizar el sistema de archivos. Solamente un dispositivo en modo bloque puede soportar un sistema de archivos montado. Los dispositivos de la red son accedidos a trav'ez de los subsistemas de red.

Hya muchos y diferentes drivers de dispositivos en el kernell de Linux, lo cual es una de sus fuerzas, pero todos comparten algos atributos en com'un:

\begin{itemize}
\item C'odigo de kernell
Los drivers son parte del kernell y, como todo otro c'odigo en el kernell, si funciona mal puede da�ar el sistema incluso perdiendo informaci'on del FS.

\item Interfaces del kernell
Los drivers deben prooveer una interfaz est'andar al kernell de Linuz o al subsistema del que son parte.

\item Mecanismos del kernell y servicios
Los drivers usan los servicios est'andar del kernell como administraci'on de memoria, manejo de interrupciones y colas de espera para operar.
   
\item Cargable
Muchos drivers pueden ser cargados a demanda como modulos del kernell cuando son necesitados y descargados cuando no son m'as usados. 'Esto hace al kernell flexible y eficiente con los recursos.

\item Configurable
Los drivers pueden ser construidos en el kernell, y se puede configurar cuales cuando se compila el kernell.

\item Din'amico
Mientras es sistema bootea y cada driver es inicializado, 'este mira por cada dispositivo que est'a controlando. No importa si el dispositivo que se est'a controlando por un dispositivo en particular no existe. En 'este caso el driver es simplemente redundante y no causa ning'un da'no aparte de ocupar un poquito de memoria del sistema.
\end{itemize}

%%%%%%%%%%%%%%%%%%%%%%%%%%%%%%%%%5

\item Administraci'on de FS: por el momento, Linux soporta 15 sistemas de archivo: ext, ext2, xia, minix, umsdos, msdos, vfat, proc, smb, ncp, iso9660, sysv, hpfs, affs y ufs. Aunque el principal, en principio (1992), era ext 'este carecia de buen rendimiento. Por eso es que fue reemplazado por ext2 en 1993. 'Este sistema, como muchos otros, es construido en la premisa que la informaci'on en los archivos es conservada en bloques. 'Estos bloques son todos de la misma longitud y, a pesar que la longitud puede variar entre diferentes sistemas EXT2, el tama'no del bloque de un sistema particular es establecido cuando es creado. Cada tama'no de archivo es redondeado hacia arriba hasta un n'umero entero de bloques. Si el tama'no de bloque es de 1024 bytes entonces un archivo de 1025 bytes ocupar'a 2 bloques. Desafortunadamente esto significa que en promedio se desperdicia la mitad de un bloque por archivo. Pero para favorecer el desempe'no del CPU 'este ineficiente \emph{tradeoff} es utilizado. No todos los bloques en el FS contienen informaci'on, algunos deben ser usados para contener la informaci'on que describe la estructura del sistema de archivos. EXT2 define la topolog'ia del sistema describiendo cada archivo en el sistema con una estrucuta de datos inodo. Un inodo describe que bloques de informaci'on dentro de un archivo ocupa al igual que los derechos de acceso del archivo, las veces que se modific'o y el tipo de archivo. Cada archivo en el FS EXT2 es descripto por un 'unico inodo y cada inodo tiene un 'unico n'umero identific'andolo. Los inodos para el sistema de archivos estan contenidos en tablas de inodos. Los directorios EXT2 son simples archivos especiales (descriptos como inodos) que contienen punteros a los inodos de sus entradas dentro del directorio.
\end{itemize}


Notar que ambos sistemas operativos poseen ciertas peque'nas similitudes como los inodos en el sistema de archivos y round robin para los procesos de usuario, pero varias caracter'isticas transforman a Linux en un sistema de producci'on a diferencia de Minix que se l'imita bastante. Por ejemplo, el tama'no m'aximo de los archivos en Minix (64MB) hace que no pueda ejecutar aplicaciones tales como una base de datos decente de hoy en d'ia, y ni hablar de archivos multimedia grandes como pel'iculas. Adem'as, la administraci'on de memoria de Minix es muy rudimentaria, sin pr'acticamente protecci'on asegurada y con un espacio de direcciones limitado a la memoria f'isica, a diferencia de Linux que posee memoria virtual a trav'ez de memoria paginada con todas las ventajas que se describieron como la proteccion y el espacio mayor de direcci'on. Es por eso que no sorprende la cantidad de actualizaciones que recibi'o Linux (al igual que su popoularidad) y el uso did'actico que se le d'a a Minix (debido a la simpleza de sus administraciones).

\subsection{Ejercicio 9}

Los dispositivos de E/S pueden dividirse en 2 categorias: dispositivos tipo 
bloque y dispositivos tipo caracter. Uno tipo bloque es el cual guarda 
infomaci'on en bloques de tama'no fijo, cada uno con su propia direcci'on. Los 
bloques comunes van del rango de los 128 bytes a los 1024 bytes. La propiedad 
esencial de este tipo de dispositivo es hacer posible la lectura o escritura de 
cada bloque independientemente de los otros. En otras palabas, instantaneamente 
el programa puede leer o escribir cualquiera de sus bloques. Un ejemplo de ellos 
son los discos.

Los dispositivos de E/S tipo caracter entrega o acepta una cadena de caracteres, 
sin respetar ninguna estructura de bloques. No es direccionable y no tiene 
ning'un tipo de operaci'on de b'usqueda. La terminal, impresoras, cintas de 
paepl, tarjetas perfordas, interfaces de red, mouses y muchos otros dipositivos 
no que no son como discos pueden verse como dispositivos tipo caracter.
Este modelo es en general suficiente para ser usado para hacer que el software 
del S.O. pueda trabajar con dispositivos de E/S independientes. El file system, 
por ejemplo, trabaja solo con dispositivos de bloques abstractos y deja la parte 
dependiente del dispositivo a software de bajo nivel llamado drivers de 
dispositivos.

El trabajo de los drivers de dispositivos es aceptar pedidos abstractos de un 
software independiente del dispositivo sobre 'el, y ver que el pedido sea 
realizado.

En MINIX, para cada clase de dispositivo de E/S, esta presente una tarea de E/S 
(I/O task) o driver de dispositivo.  Estos drivers son procesos terminados, cada 
uno con su propio estado, mapa de memoria y m'as. Estos driverse se comunican 
entre si, si es necesario,  con el file system usando el mecanismo de pasaje de 
mensajes estandar usado por todos los procesos de MINIX. Adem'as, cada driver de 
dispositivo esta escrito en un solo archivo fuente, como floppy.c o clock.c. La 
'unica diferencia entre los drivers y otros procesos es que los primeros estas 
linkeados con el kernel, y comparten todo el espacio de de direccionamiento.

El programa principal para cada driver de dispositivo tipo bloque es 
estructuralmente el mismo. Posee un ciclo infinito, el cual espera la llegada de 
un mensaje invocandolo. Si esto sucede, se llamar� a la operaci�n del tipo 
entregado por el mensaje.

La estructura es la siguiente:     

\begin{verbatim}

message mess;                       /* buffer del mensaje */

void io_task() {
    initialize();                   /* solo se inicializa una vez */
    while(TRUE){
        receive(ANY, &mess);        /* espera a un pedido de trabajo */
        caller = mess.source;       /* proceso desde donde vino el mensaje */
        switch(mess.type){
            case READ: rcode = dev_read(&mess);break;
            case WRITE: rcode = dev_write(&mess);break;
            /* Other cases go here, e.g., OPEN, CLOSE, IOTCTL */
            default: rcode = ERROR;
        }
        mess.type = TASK_REPLY;
        mess.status = rcode;        /* respuesta */
        send(caller, &mess);        /* Se envia respuesta al proceso llamador */
    }                                  
}

\end{verbatim}
                                    
Cuando el sistema inicia, cada uno de los drivers se inicializa para definir 
datos internos como tablas y similares cosas. Cada task intenta recibir un 
mensaje. Cuando alguno llega, se graba la identidad del llamador, y se ejecuta 
un procedimiento en particular para realizar el trabajo. Luego de finalizado, 
una respuesta es enviada al llamador, y el task vuelve a el tope del ciclo 
esperando un nuevo pedido.


Para generar un driver de dispositivo tipo bloque hay que hacer lo siguiente:

\begin{enumerate}
\item hay que alterar el /usr/include/minix/com.h. Este archivo es el que guarda el 
id de cada uno de los I/O task. Este id es llamado "major number" del 
dispositivo. Este n'umero especifica la clase de dispositivo, como los floppy, 
disco r'igido, o terminal. 
Todos los dispositivos con el mismo "major number" comparten el mismo codigo de 
driver dentro del S.O.

\item luego hay que alterar /usr/include/minix/config.h para determinar cuales son 
los dispositivos con los cuales el kernel va a ser compilado. Incluir todos 
estaria mal. Esto es para que sea m'as facil excluir el nuevo dispositivo. 

\item alterar el /usr/include/minix/const.h Se encarga de las constantes usadas por 
el MINIX. Se la usa para agregar en nuevo proceso a las NR\_TASKS. Setea el 
numero de las tareas predefinidas del kernel incluyendo los drivers de 
dispositivo. 

\item /usr/src/kernel/proto.h Hay que incluir el header o prototipo de la llamada 
al driver. Eso es, predefinir el punto de entrada del nuevo driver. Este punto 
es la primer funci'on que se ejecuta en el driver, como el main() es el punto de 
entrada de un archivo de C.
En el ejemplo, el task va a manejar un disco, se llamara en tal caso, 
\verb0disco_task0.

\begin{verbatim}
PUBLIC _PROTOTYPE( void disco_task, (void) );
\end{verbatim}

\item /usr/src/kernel/table.c En este archivo debe existir una entrada para las 
tasks. En ella se especifica el tama'no del stack que va a usar el nuevo driver. 

\begin{verbatim}
#define DISCO_STACK (4 * SMALL_STACK * ENABLE_DISCO)
\end{verbatim}

Adem'as se debe agregar dentro del struct tasktab una l'inea donde se relacionan 
el task(el nombred del driver), la pila y el programa del driver propiamente 
dicho en ese orden.

\begin{verbatim}
#if ENABLE_DISCO

{ disco_task, DISCO_STACK , "DISCO" },

#endif                 
\end{verbatim}

Donde \emph{DISCO} es el nombre del proceso.

\item \verb0/usr/src/kernel/table.c0 Tambi'en hay que agregar a este archivo una entrada 
para el nombre de la task en particular. Hay que agregar un archivo especial, en 
nuestro caso "DISCO" en /dev y ligar el proceso del nuevo driver con 'el. 
La nueva l'inea se agrega en el struct dmap. (como muesta se agrega el 
encabezado que aparece en el fuente ).

\begin{verbatim}
?    Open        Read/Write  Close       Task #      Device  File
-    ----        ----------  -----       -------     ------  ----

DT(1, dev_opcl,   call_task,  dev_opcl,   DISCO)      /* 7 = /dev/disco */
\end{verbatim}


\item Hay que generar el driver mismo. Depende de que es lo que maneje, la 
estructura del mismo. Para el ejemplo citado, el archivo se llamar� disco.c y se 
encontrar� en la ruta \verb0/usr/src/kernel/disco.c0

\item Es necesario alterar el Makefile del kernel para que el nuevo driver sea 
compilado con 'el.                        

\item Hay que generar el iNodo:

\verb0mknod /dev/disco Flags MayorNumber MinorNumber\verb0

\end{enumerate}

\subsection{Ejercicio 10: Modificaci'on de c'odigos}
\begin{description}
\item[a.] Modifique el "scheduler"\ original del MINIX para el nivel de usuarios.
\item[b.] Modifique la administraci'on de memoria original del MINIX
\end{description}
En ambos casos deber'a describir en el informe cu'ales fueron las decisiones tomadas, cu'ales fueron las espectativas y cu'ales fueron los resultados obtenidos e informar el juego de programas utilizados con los cuales se lleg'o a alguna conclusi'on. (test o pruebas mencionados en Forma de entrega).

\subsubsection{Modique el "scheduler"\ original del MINIX para el nivel de usuarios}
\subsubsubsection{Decisiones tomadas}

La administraci'on de procesos de Minix divide a los procesos en tres categor'ias: procesos de usuario, procesos SERVER (como ser MM, o FS) y procesos TASK (kernel). Asimismo, para los procesos de usuario, utiliza una administraci'on $Round-Robin$, y un proceso s'olo podr'a perder el recurso procesador por alguna de las tres siguientes razones:

\begin{itemize}
\item El proceso finaliza.
\item Se bloquee por una operaci'on de E/S o sincronizaci'on.
\item El uso exceda un cierto $Quantum$.
\end{itemize}

Decidimos modificar la administraci'on de los procesos de usuario para que cambie de $Round-Robin$ a $FIFO$,

\subsubsubsection{Expectativas}

Como estamos permitiendo un uso m'as prolongado y continuo del recurso procesador, es de esperar que los procesos de alto uso de CPU se vean beneficiados, quiz'as en detrimento de procesos que tengan, en cambio, mayor uso de E/S, ya que 'estos se bloquear'an r'apidamente y estar'an un largo tiempo en espera del recurso.

\subsubsubsection{Resultados}

Nuestra propuesta es modificar el archivo \textbf{/usr/src/kernel/clock.c}. De la siguiente manera: si esta definido UBA\_FCEN, entonces en la funci'on $do\_clocktick$ no se ejecutar'a el c'odigo que realiza la selecci'on de un proceso de usuario nuevo, en el caso en que no haya ni procesos SERVER ni TASK para elegir y adem'as que el proceso anterior haya excedido su $Quantum$.

El c'odigo que no se ejecuta si UBA\_FCEN est'a definido est'a en el Ap'endice A.

Se generaron dos procesos de prueba \textbf{highCPU.c} y \textbf{lowCPU.c}. El primero consiste en la repetici'on de un ciclo que contiene a su vez otro ciclo, en total clica aproximadamente 5000050000 veces. El segundo consiste en la repetici'on de un ciclo 10000 veces. Ambos al terminar muestran por pantalla la etiqueta: "Final de highCPU"\ y "Final de lowCPU"\ respectivamente.

A continuaci�n se muestran los resultados obtenidos con la administraci'on $Round-Robin$:

\begin{verbatim}
    # cd /usr/grupo8/ej10
    # ls
    highCPU highCPU.c lowCPU lowCPU.c
    #./highCPU & ./lowCPU 
    Final de lowCPU
    # Final de highCPU
    #
    #./lowCPU & ./highCPU
    Final de lowCPU
    # Final de highCPU
\end{verbatim}

Si cambiamos a la administraci'on a $FIFO$ obtenemos los siguientes valores:

\begin{verbatim}
    # cd /usr/grupo8/ej10
    # ls
    highCPU highCPU.c lowCPU lowCPU.c
    #./highCPU & ./lowCPU 
    Final de lowCPU
    # Final de highCPU
    #
    #./lowCPU & ./highCPU
    Final de highCPU
    # Final de lowCPU
\end{verbatim}

Con la administraci'on $Round-Robin$ lowCPU siempre termina primero. Debido a que hace poco uso del procesador (menos que highCPU) y adem'as tiene su "\ cuota"\ de procesador correspondiente debido al tipo de administraci'on. Esto hace que termine de ejecutar antes que highCPU.

Con la administraci'on $FIFO$ termina primero el que obtiene el recurso procesador primero. Ninguno de los procesos se bloquea, dando lugar a ejecutar al otro, ya que no realizan E/S.

\subsubsection{Modifique la administraci'on de memoria original del MINIX}

\subsubsubsection{Decisiones tomadas}
MINIX utiliza administraci'on de memoria particionada variable, con $primer\ zona$ (first fit) como algoritmo de selecci'on. Se cambi'o el procedimiento de alocaci'on de memoria para que utilice el algoritmo de selecci'on $mayor\ zona$.

El c'odigo modificado se encuentra en el Ap'endice A.

\subsubsubsection{Expectativas}
'Estas administraciones fueron vistas durante el curso. Tanto la administraci'on $mayor\ zona$ como la administraci'on "golosa" de $first\ fit$ tienen casos en la que resultan buenas y casos en las que no. Con lo cual no se espera alg'un comportamiento en particular, salvo porque se deber'ia de estar tomando siempre el segmento de mayor tama'no.

\subsubsubsection{Resultados}
Para realizar las modificaciones, cambiamos la secci'on de alocaci'on de memoria que ocurre en la funci'on $alloc\_mem$ del archivo \textbf{/usr/src/mm/alloc.c}. En ella se hace un while recorriendo los segmentos libres en la memoria del sistema en busca del de mayor tama'no. Cuando se encuentra (en el caso en que cubra el requerimiento) nos quedamos con ese lugar, caso contrario se devuelve un mensaje de error (Ver Ap'endice).

Para poder ver los resultados agregamos dentro de la funci'on c'odigo que nos muestra el tama'no de los segmentos libres antes y despu'es de la alocaci'on. Para que dicho c'odigo se ejecute debe estar definida la variable DEBUGG. Dicha variable se encuentra en \textbf{/usr/include/minix/config.h}.

Usamos el programa de test 4000Clicks.c que puede encontrarse en la carpeta \textbf{/usr/grupo8/ej10}:
\begin{verbatim}
    static char datos [4000*256]; /* reserva 1000k */
    int main ( void)
    {
    return 0 ;
    }
\end{verbatim}

Este programa lo 'unico que hace es reservar 4000 clicks (de 256 Bytes) de memoria que es equivalente a 1000 KB.
Lo ejecutamos y observamos lo siguiente:

\begin{verbatim}
    # ls
    Se quiere reservar 306 espacios de memoria
    1936 514 67 5860
    Los segmentos quedaron de la siguiente manera
    1936 514 67 5554
    Se quiere reservar 440 espacios de memoria
    1936 514 67 5860
    Los segmentos quedaron de la siguiente manera
    1936 514 67 5420
    4000Clicks 4000Clicks.c
    # ./4000Clicks
    Se quiere reservar 306 espacios de memoria
    1936 514 67 5860
    Los segmentos quedaron de la siguiente manera
    1936 514 67 5554
    Se quiere reservar 4514 espacios de memoria
    1936 514 67 5860
    Los segmentos quedaron de la siguiente manera
    1936 514 67 1346
    #
\end{verbatim}

Notar que el cambio fue el deseado, ya que al pedir memoria siempre se ocupa el segmento de mayor tama'no. Usando la administraci'on anterior se ubiese ocupado el primer bloque.

Aclaraci'on: El click es la unidad b'asica de tama'no de memoria. Si el procesador es Intel est'a definida como 256 Bytes y var'ia para otros procesadores. Estos $define$ se encuentran en $const.h$.
Para poder ver como se va modificando la memoria hay que bootear con la imagen $imagMemChck$ que est'a en el directorio \textbf{/minix}. %Las otras im'agenes tambi'en tienen la modificaci'on en la administraci'on de memoria pero el "debug"\ est'a \"apagado".

\subsection{Ejercicio 11}
Se realizaron los siguientes cambios

\verb�/usr/include/minix/callnr.h�

Se increment'o en 1 el define NCALLS
Se agreg� \verb0#define LLAMSISTEMA0 con el nro correspondiente

\verb�/usr/src/mm/proto.h�
Se agrega linea:

\verb�_PROTOTYPE( int do_llamsistema, (void)	);�


\verb�/usr/src/mm/table.c�
Se agrega linea:

\verb�do_llamsistema, 	/\textasteriskcentered 79 = LLAMSISTEMA \textasteriskcentered/�


\verb�/usr/src/fs/table.c�
Se agrega linea:

\verb�no_sys, 		/\textasteriskcentered 79 = LLAMSISTEMA \textasteriskcentered/�


Se implementa System Call en:
\verb�/usr/src/mm/llamsistema.c�
Se actualiza Makefile de \verb�/usr/src/mm� para incluirlo.

Se agrega la linea al archivo
\verb�/usr/include/minix/com.h�

\verb�#define OPC_NEWCALL		mi_i1�

Para renombrar el par'ametro que se usa para la elecci'on de la consulta que se hace por medio del System Call.

Se recompila kernel

Ejecutar

\begin{verbatim}
# /usr/src/tools/make install
# /usr/src/tools/make hdboot
\end{verbatim}



\emph{Nota: no se us'o el nombre pedido x la c'atedra (newcall), ya que este fue usado para otras pruebas. Se realiz'o una version del getpid desde MM y desde FS, sendas llamadas a sistema con nombres newcall y othercall respectivamente.}

\subsubsection{Pruebas}

En imagen minix para ejercicio 11

Fuente: \verb0/usr/ej11/llam.c0

Ejecutable: \verb0/usr/ej11/llam0

Modo de prueba

\begin{verbatim}
./llam [opcion]
\end{verbatim}

Opcion es del 1 al 5

\begin{enumerate}
\item Entrega el pid del programa
\item Entrega el pid del padre, que, al estar implementado en MM, es el pid del mm o sea 0
\item Entrega el puntero al segmento text, en hexadecimal
\item Entrega el puntero al segmento data, en hexadecimal
\item Entrega el puntero al segmento stack, en hexadecimal
\end{enumerate}

Para las opciones 3, 4 y 5, para verificar la correctitud de la misma se agreg'o un ciclo while para evitar que el programa termine. Al ejecutar el programa con algunas de estas opciones, al presionar F2 veremos la informacion de los segmentos correspondientes. Para matar al proceso, abrimos otra consola, ejecutamos ps para ver el nro del mismo y con el comando \verb0kill [nro proc]0 lo eliminamos.

En el mismo directorio el newcall.c y el othercall.c son source para probar dichas funciones
\begin{verbatim}
./newcall
./othercall
\end{verbatim}

Tienen la misma funcionalidad que \verb0./llam 10
La funci'on \verb0NEWCALL0 utiliza la informaci'on del Memory Manager (igual que \verb0LLAMSISTEMA0) para averiguar el pid y \verb0OTHERCALL0, el File System.

\subsection{Ejercicio 12}
Para el dise�o de sem�foros se eligi'o la esctructura que se ve en el archivo semaf.h. El typedef semaforo se cre'o para que el usuario del sem�foro no posea m�s informaci'on del mismo que el id. La estructura semaf es la que guarda toda la informaci'on sensible para el manejo de los sem�foros y el arreglo semaforos hace las veces de objeto contenedor de los mismos. El nombre del sem�foro se usa para identificarlo, y tiene un largo m�ximo dado por el largo de \verb�M3_STRING�, que es un arreglo de char de 15 posiciones. Se tom'o esta decisi'on porque fue mas simple manejar el nombre del sem�foro a trav�s de un arreglo que con punteros. El flag semafEnUso sirve para saber si un semaforo tiene asignado alg�n proceso. Estos se agregan en la lista procEnUso, en el primer lugar vacio que encuentren. El campo valor, indica el valor del sem�foro ( ya que son sem�foros contadores ). La cola de procesos bloqueados est� implementada sobre un arreglo y 2 apuntadores, al inicio y al fin de la lista respectivamente. La lista va haciendose circular por el arreglo, una vez que se llega al final del mismo, vuelve a comenzar.

El usuario posee las siguientes funciones para el manejo de sem�foros:

\verb�semaforo crear_sem(char* nombre, int valor)� : A partir de un nombre y un valor crea un sem�foro, le asigna nombre, valor y el proceso que lo cre'o. Retorna un ''semaforo'' (int). Si el nombre ya exist�a en la lista de sem�foros, asigna al proceso creador e ignora el valor entregado. Si no hay m�s sem�foros disponibles retorna mensaje de error.

\verb�int p_sem(semaforo x)� : Ejecuta P al sem�foro x. Decrementa el valor del sem�foro. Si el valor resulta quedar en 0 o menor, el proceso se bloquea, se agrega a la cola de bloqueados. Si el sem�foro no pertenec�a al proceso se retorna mensaje de error.

\verb�int v_sem(semaforo x)� : Ejecuta V al sem�foro x. Incrementa el valor del sem�foro. Si el valor resulta quedar en 0 o menor, el primer proceso de la cola de bloqueados se libera. Si el sem�foro no pertenec�a al proceso se retorna mensaje de error.

\verb�int liberar_sem(semaforo x)� : Dado un sem�foro x libera a todos los procesos bloqueados en �l, borra los procesos de la lista e inicializa al sem�foro. Este m�todo viola la exclusi'on mutua.

\verb�void inicializar()� : Vacia las listas de procesos, flags, valor y nombre de todos los sem�foros

\vspace{1cm}

\subsubsubsection{Cambios para ejercicio 12}

\verb�/usr/include/minix/callnr.h�

\begin{list}{}{}
\item Se increment'o en 5 el define \verb�NCALLS� ( Se definieron 5 llamadas a sistema )
\item Se agreg'o \verb�#define CREAR_SEM� con el nro correspondiente
\item Se agreg'o \verb�#define P_SEM� con el nro correspondiente
\item Se agreg'o \verb�#define V_SEM� con el nro correspondiente
\item Se agreg'o \verb�#define LIBERAR_SEM� con el nro correspondiente
\item Se agreg'o \verb�#define INIT_ALL_SEM� con el nro correspondiente
\end{list}

\verb�/usr/src/mm/proto.h�

Se agrega linea:

\begin{verbatim}
_PROTOTYPE( int do_crear_sem, (void)	);
_PROTOTYPE( int do_p_sem, (void)		);
_PROTOTYPE( int do_v_sem, (void)		);
_PROTOTYPE( int do_liberar_sem, (void)	);
_PROTOTYPE( int do_init_all_sem, (void)	);
\end{verbatim}

\verb�/usr/src/mm/table.c�

Se agregan lineas:

\begin{verbatim}
do_crear_sem, 		/* 80 = CREAR_SEM		*/
do_p_sem, 			/* 81 = P_SEM 			*/
do_v_sem, 			/* 82 = V_SEM			*/
do_liberar_sem, 	/* 83 = LIBERAR_SEM		*/
do_init_all_sem, 	/* 84 = INIT_ALL_SEM	*/
\end{verbatim}

\verb�/usr/src/fs/table.c�

Se agregan lineas:

\begin{verbatim}
no_sys, 			/* 80 = CREAR_SEM		*/
no_sys, 			/* 81 = P_SEM 			*/
no_sys, 			/* 82 = V_SEM			*/
no_sys, 			/* 83 = LIBERAR_SEM		*/
no_sys, 			/* 84 = INIT_ALL_SEM	*/
\end{verbatim}

Se define estructura del sem�foro en:
\verb�/usr/src/mm/semaf.h� (archivo source)

Se implementan system call en:
\verb�/usr/src/mm/semaf.c� (archivo source)

Se actualiza Makefile de \verb�/usr/src/mm� para incluirlos.

Se crea
\verb�/usr/include/minix/constsemaf.h�  (archivo source)
Posee unos defines para renombrar los par�metros del mensaje

Se actualiza archivo system.c para agregar los system call encargados de bloquear y desbloquear un proceso.
\verb�/usr/src/kernel/system.c� (archivo source)

Se recompila kernel

Ejecutar

\begin{verbatim}
# /usr/src/tools/make install
# /usr/src/tools/make hdboot
\end{verbatim}

Se crean archivos para poder llamar el system call desde una librer�a y cumplir el standard posix:

\begin{verbatim}
/usr/src/lib/syscall/crear_sem.s (archivo source)
/usr/src/lib/syscall/p_sem.s (archivo source)
/usr/src/lib/syscall/v_sem.s (archivo source)
/usr/src/lib/syscall/liberar_sem.s (archivo source)
/usr/src/lib/syscall/init.s (archivo source)
\end{verbatim}
Se actualiza Makefile de \verb�/usr/src/lib/syscall� para incluirlo.

\verb�/usr/src/lib/posix/_sem.c� (archivo source)
Se actualiza Makefile de \verb�/usr/src/lib/posix� para incluirlo.

Se crean archivos para poder manejar como funci'on los system call al kernel:
\begin{verbatim}
/usr/src/lib/syslib/sys_block.c (archivo source)
/usr/src/lib/syslib/sys_unblock.c (archivo source)
\end{verbatim}
Se actualiza Makefile de \verb�/usr/src/lib/syslib� para incluirlo.

\verb�/usr/include/minix/syslib.h�

Se agregan lineas:
\begin{verbatim}
_PROTOTYPE( int sys_block, (int_proc)	);
_PROTOTYPE( int sys_block, (int_proc)	);
\end{verbatim}

Ejecutar

\begin{verbatim}
# /usr/src/lib/make all
# /usr/src/lib/make install
\end{verbatim}

\subsubsection{Pruebas ejercicio 12}
En imagen minix para ejercicio 12

Directorio \verb�/usr/ej12�:

Est�n implementados productor consumidor y secuencia de procesos A-B-B-A-C

\vspace{1cm}
\subsubsubsection{Productor-Consumidor}

Ejecutar

\begin{verbatim}
# ./init --> inicializa sem�foros
# ./prod-cons --> define sem�foros x e y para pruebas productor consumidor

# ./productor  > /dev/null &   --> ejecuta productor en 2do plano
# ./consumidor > /dev/null &  --> ejecuta consumidor en 2do plano
\end{verbatim}

Ejecutar indistintamente productor y consumidor. Cada vez que se produzca algo se mostrar� en consola ''Produce'' y cada vez que se consuma, ''Consume''

Para finalizar la prueba ejecutar:

\verb�# ./libsem�

que libera todos los procesos bloqueados

\vspace{1cm}
\subsubsubsection{Secuencia A-B-B-A-C}

Ejecutar

\begin{verbatim}
# ./init --> inicializa sem�foros
# ./sem --> define todos los sem�foros necesarios
\end{verbatim}

Los procesos involucrados son ./A ./B o ./C . Cada vez que se ejecutan se muestra en consola ''este es A'', ''este es B'' o ''este es C'' respectivamente.

\begin{verbatim}
# ./A > /dev/null &   --> ejecuta A en 2do plano
# ./B > /dev/null &   --> ejecuta B en 2do plano
# ./C > /dev/null &  --> ejecuta C en 2do plano
\end{verbatim}

Hay un script ( pru.sh ) que tiene un ejemplo:

\verb� ./A & ./A & ./A & ./B & ./B & ./B & ./C & ./C & ./C & >  /dev/null &�

Que deber�a mostrar lo siguiente:
\begin{verbatim}
este es A
este es B
este es B
este es A
este es C
este es A
este es B
\end{verbatim}

Ejecutando el comando

\verb�# ps�

Podemos corroborar que existen 2 procesos C que est�n bloqueados.

Ejecutar indistintamente productor y consumidor. Cada vez que se produzca algo se mostrar� en consola ''Produce'' y cada vez que se consuma, ''Consume''

Para finalizar la prueba ejecutar:

\verb�# ./libsem�

que libera todos los procesos bloqueados


\clearpage
\section{Apendice}
\subsection{Fuentes del ejercicio 11}

llamsistema.c

\begin{verbatim}
#include "mm.h"
#include <minix/callnr.h>
#include <signal.h>
#include "mproc.h"
#include <stdlib.h>
#include <minix/com.h>
#include <minix/type.h>

PUBLIC int do_llamsistema( void ) {

	register struct mproc *rmp = mp;
	
	register int r;

	switch( mm_in.OPC_NEWCALL ) {

		case 1: /* pid */

			r = mproc[who].mp_pid;
			break;

		case 2: /* ppid */

			break;

		case 3: /* text */

			r = (int) mproc[who].mp_seg[T].mem_phys;
			break;

		case 4: /* data */
			
			r = (int) mproc[who].mp_seg[D].mem_phys;
			break;

		case 5: /* stack */

			r = (int) mproc[who].mp_seg[S].mem_phys;
			break;

		default:
			
			break;

	}

	return r;

}

\end{verbatim}

\_llamsistema.c

\begin{verbatim}
#include <lib.h>
#include <unistd.h>
#include <minix/com.h>
#define llamsistema _llamsistema

PUBLIC int llamsistema( int opcion ) {

	message m;

	switch( opcion ) {

		case 1:	/* pid */
		case 3: /* text */
		case 4: /* data */
		case 5: /* stack */

			m.OPC_NEWCALL = opcion;
	
			return(_syscall( MM, LLAMSISTEMA, &m ));

		case 2: /* ppid */
	
			_syscall( MM, LLAMSISTEMA, &m );
			return m.m2_i1;

		default:

			return -1;

	} 
}
\end{verbatim}

llamsistema.s

\begin{verbatim}
.sect .text
.extern __llamsistema
.define _llamsistema

.align 2

_llamsistema:
	jmp __llamsistema
\end{verbatim}

\subsection{Fuentes del ejercicio 12}

constsemaf.h

\begin{verbatim}
#define PROC1 		m1_i1
#define NOMBRE_SEM	m3_ca1
#define VALOR		m3_i1
#define SEMAFORO	m1_i2
\end{verbatim}

crear\_sem.s

\begin{verbatim}
.sect .text
.extern __crear_sem
.define _crear_sem

.align 2

_crear_sem:
	jmp __crear_sem
\end{verbatim}

init.s

\begin{verbatim}
.sect .text
.extern __inicializar
.define _inicializar

.align 2

_liberar_sem:
	jmp __inicializar
\end{verbatim}

liberar\_sem.s

\begin{verbatim}
.sect .text
.extern __liberar_sem
.define _liberar_sem

.align 2

_liberar_sem:
	jmp __liberar_sem
\end{verbatim}

p\_sem.s

\begin{verbatim}
.sect .text
.extern __p_sem
.define _p_sem

.align 2

_p_sem:
	jmp __p_sem
\end{verbatim}

semaf.c

\begin{verbatim}
#include "mm.h"
#include <minix/callnr.h>
#include <signal.h>
#include "mproc.h"
#include <stdlib.h>
#include "semaf.h"
#include <minix/constsemaf.h>
#include <minix/com.h>
#include <minix/type.h>
#include <string.h>

FORWARD _PROTOTYPE( int do_is_sem, (void)				);
FORWARD _PROTOTYPE( int do_val_sem, (void)				);
FORWARD _PROTOTYPE( int do_get_next_bloq_proc, (void)			);
FORWARD _PROTOTYPE( void do_add_bloq_proc, (int)			);
FORWARD _PROTOTYPE( void do_init_sem, (void)				);

/*===========================================================================*
 *			do_crear_sem					     *
 *===========================================================================*/

PUBLIC int do_crear_sem(void) {

	register struct semaf *sp = semaforos;

	semaforo s;

	pid_t procID		= mproc[who].mp_pid;
	char* nombre		= mm_in.NOMBRE_SEM;
	int valor		= mm_in.VALOR;

	int i, j;

	i = 0;

	/* printf("este es el proc en crear %d\n",procID); */


	for(i=0;i<MAX_SEM;i++) {
		/* Busco si el nombre corresponde a un semaforo ya creado */
		if (!strcmp(semaforos[i].nombre, nombre)) {
			/* printf("semaforo encontrado!!\n"); */
			break;
		}
	}

	/* i se incrementa en uno mas, no se xq, entonces lo decremento */

	if ( i < MAX_SEM ) {
		/* Si el semaforo ya existe lo selecciono */
		s = i;

	} else {
		/* Sino busco el primer semaforo sin uso */
		for(i=0;i<MAX_SEM;i++) {
			if (semaforos[i].semafEnUso == 0) {

				s = i;
				/* Defino el semaforo */

				strcpy(semaforos[s].nombre, nombre);
				semaforos[s].valor = valor;
				semaforos[s].semafEnUso = 1;
				semaforos[s].cant_proc  = 0;
				semaforos[s].inicio_cola_bloq = 0;
				semaforos[s].fin_cola_bloq    = 0;

				break;

			}
		}
	}


	if (i == MAX_SEM) {

		/* si no hay semaforos disponibles retorno error */
		return -1;

	} else {

		semaforos[s].cant_proc++;

		/* Asigno al proceso al primer lugar vacio de la lista de procesos */
		for(j=0;j<MAX_PROC;j++) {
			if (semaforos[s].procEnUso[j] == 0) {

				semaforos[s].procEnUso[j] = procID;
				break;

			}
		}

		return s;

	}

}


/*===========================================================================*
 *			do_is_sem		    			     *
 *===========================================================================*/

PRIVATE int do_is_sem(void) {

	register int i;

	register struct semaf *sp = semaforos;

	pid_t procID = mproc[who].mp_pid;
	semaforo s = mm_in.SEMAFORO;

	/* printf("el proc es: %d\n", procID); */

	for(i=0;i<MAX_PROC;i++) {
		/*
		printf("proc en uso actual: %d\n", semaforos[s].procEnUso[i]);
		*/
		if (semaforos[s].procEnUso[i] == procID) {
			return 1;
		}
	}

	return 0;

}

/*===========================================================================*
 *				do_p_sem					     							 *
 *===========================================================================*/

PUBLIC int do_p_sem(void) {

	message m;

	register struct semaf *sp = semaforos;

	register int proc_nr;
	register semaforo s = mm_in.SEMAFORO;

       /**
	 * Calculo la posicion del proceso en la tabla.
	 * Esta es el puntero al proceso menos el puntero a la lista
	 * de procesos.
 	 */

	proc_nr = (int) (mp - mproc);

	if (do_is_sem()) {

		/* decremento el valor del semaforo */
		semaforos[s].valor--;

		if(do_val_sem() < 0 ) {

			/* Agrego el proceso a bloqueados */
			do_add_bloq_proc(proc_nr);

			/*  debo bloquear el proceso */
			sys_block(proc_nr);

		}

		return 0;

	} else {

		return -1;

	}

}

/*===========================================================================*
 *				do_v_sem					     							 *
 *===========================================================================*/

PUBLIC int do_v_sem(void) {

	message m;

	register struct semaf *sp = semaforos;

	register int proc_nr;
	semaforo s = mm_in.SEMAFORO;

	if (do_is_sem()) {

		/* incremento el valor del semaforo */
		semaforos[s].valor++;

		if(do_val_sem() <= 0 ) {

			/* Busco si existe algun proceso bloqueado */
			proc_nr = do_get_next_bloq_proc();

			if (proc_nr > 0) {

				sys_unblock(proc_nr);

			}

		}

		return 0;

	} else {

		return -1;

	}

}

/*===========================================================================*
 *			do_val_sem					     *
 *===========================================================================*/

PRIVATE int do_val_sem(void) {

	register struct semaf *sp = semaforos;

	register semaforo s = mm_in.SEMAFORO;

	return semaforos[s].valor;

}

/*===========================================================================*
 *			do_liberar_sem   				     *
 *===========================================================================*/

PUBLIC int do_liberar_sem(void) {

	message m;

	register struct semaf *sp = semaforos;

	register int proc_nr;

	pid_t procID = mproc[who].mp_pid;
	semaforo s = mm_in.SEMAFORO;

	int i;

	if (do_is_sem()) {

		semaforos[s].cant_proc--;

		if (semaforos[s].cant_proc > 0) {
		/* si el semaforo posee procesos bloqueados los desbloquea */

			for(i=semaforos[s].inicio_cola_bloq;i<=semaforos[s].inicio_cola_bloq;i++) {

				proc_nr = semaforos[s].procBloqueados[i];
				sys_unblock(proc_nr);

			}

		/* Luego elimino todos los procesos asociados referencia al proceso */

			for(i=0;i<MAX_PROC;i++) {

				semaforos[s].procEnUso[i] = 0;

			}

		}

		/* finalmente inicializo el semaforo */

		do_init_sem();

		return 0;

	} else {

		return -1;

	}

}

/*===========================================================================*
 *			do_get_next_bloq_proc				     *
 *===========================================================================*/

PRIVATE int do_get_next_bloq_proc(void) {

	register struct semaf *sp = semaforos;

	semaforo s = mm_in.SEMAFORO;

	register int r = 0;

	if (semaforos[s].inicio_cola_bloq != semaforos[s].inicio_cola_bloq) {

		/* si existe algun proceso bloqueado lo elijo, y adelanto el inicio de la cola */

		semaforos[s].cant_proc--;

		/* primer proceso bloqueado */
		r = semaforos[s].procBloqueados[semaforos[s].inicio_cola_bloq];

		semaforos[s].inicio_cola_bloq++;
    		semaforos[s].inicio_cola_bloq %= MAX_PROC;

	}
	/* sino retorna 0 */

	return r;

}

/*===========================================================================*
 *			do_add_bloq_proc				     *
 *===========================================================================*/

PRIVATE void do_add_bloq_proc(int proc_nr) {

	register struct semaf *sp = semaforos;

	semaforo s = mm_in.SEMAFORO;

	semaforos[s].cant_proc++;

	semaforos[s].procBloqueados[semaforos[s].fin_cola_bloq] = proc_nr;

	semaforos[s].fin_cola_bloq++;
	semaforos[s].fin_cola_bloq %= MAX_PROC;

}


/*===========================================================================*
 *				do_init_sem				   									 *
 *===========================================================================*/

PRIVATE void do_init_sem(void) {

	register struct semaf *sp = semaforos;

	register semaforo s = mm_in.SEMAFORO;

	semaforos[s].semafEnUso = 0;

	semaforos[s].valor 	= 0;
	semaforos[s].cant_proc 	= 0;

	semaforos[s].inicio_cola_bloq	= 0;
	semaforos[s].fin_cola_bloq	= 0;

}



/*===========================================================================*
 *				do_init_all_sem											     *
 *===========================================================================*/

PUBLIC int do_init_all_sem(void) {

	int i, j;
	register struct semaf *sp = semaforos;

	for(i=0;i<MAX_SEM;i++) {

		strcpy(semaforos[i].nombre,"");
		semaforos[i].semafEnUso			= 0;
		semaforos[i].valor 			= 0;
		semaforos[i].cant_proc 			= 0;
		semaforos[i].inicio_cola_bloq		= 0;
		semaforos[i].fin_cola_bloq		= 0;


		for(j=0;j<MAX_PROC;j++) {

			semaforos[i].procBloqueados[j]	= 0;
			semaforos[i].procEnUso[j] 	= 0;

		}

	}

	return 0;
}
\end{verbatim}

semaf.h

\begin{verbatim}
#define MAX_SEM 	100	
#define MAX_PROC 	100

#include <minix/type.h>

typedef int semaforo;

struct semaf {

	char nombre[M3_STRING];

	int semafEnUso; /* 0 o 1 */

	int valor;
	int cant_proc;

	int procEnUso[MAX_PROC];

	int inicio_cola_bloq;
	int fin_cola_bloq;
	int procBloqueados[MAX_PROC];

};

struct semaf semaforos[MAX_SEM];
\end{verbatim}

semaforo.h

\begin{verbatim}
/* El header <semaforos.h> contiene constantes para la definicion de semaforos */

#ifndef _SEMAFOROS_H_
#define _SEMAFOROS_H_

#define ERROR -1
typedef int semaforo;

_PROTOTYPE( semaforo crear_sem, (char* nombre, int valor) 	);

_PROTOTYPE( int p_sem, (semaforo) 				);

_PROTOTYPE( int v_sem, (semaforo)				);

_PROTOTYPE( int liberar_sem, (semaforo)				);

_PROTOTYPE( void inicializar, (void) 				);

#endif 
\end{verbatim}

system.c

\begin{verbatim}
/* This task handles the interface between file system and kernel as well as
 * between memory manager and kernel.  System services are obtained by sending
 * sys_task() a message specifying what is needed.  To make life easier for
 * MM and FS, a library is provided with routines whose names are of the
 * form sys_xxx, e.g. sys_xit sends the SYS_XIT message to sys_task.  The
 * message types and parameters are:
 *
 *   SYS_FORK	 informs kernel that a process has forked
 *   SYS_NEWMAP	 allows MM to set up a process memory map
 *   SYS_GETMAP	 allows MM to get a process' memory map
 *   SYS_EXEC	 sets program counter and stack pointer after EXEC
 *   SYS_XIT	 informs kernel that a process has exited
 *   SYS_GETSP	 caller wants to read out some process' stack pointer
 *   SYS_TIMES	 caller wants to get accounting times for a process
 *   SYS_ABORT	 MM or FS cannot go on; abort MINIX
 *   SYS_FRESH	 start with a fresh process image during EXEC (68000 only)
 *   SYS_SENDSIG send a signal to a process (POSIX style)
 *   SYS_SIGRETURN complete POSIX-style signalling
 *   SYS_KILL	 cause a signal to be sent via MM
 *   SYS_ENDSIG	 finish up after SYS_KILL-type signal
 *   SYS_COPY	 request a block of data to be copied between processes
 *   SYS_VCOPY   request a series of data blocks to be copied between procs
 *   SYS_GBOOT	 copies the boot parameters to a process
 *   SYS_MEM	 returns the next free chunk of physical memory
 *   SYS_UMAP	 compute the physical address for a given virtual address
 *   SYS_TRACE	 request a trace operation
 *
 * Message types and parameters:
 *
 *    m_type       PROC1     PROC2      PID     MEM_PTR
 * ------------------------------------------------------
 * | SYS_FORK   | parent  |  child  |   pid   |         |
 * |------------+---------+---------+---------+---------|
 * | SYS_NEWMAP | proc nr |         |         | map ptr |
 * |------------+---------+---------+---------+---------|
 * | SYS_EXEC   | proc nr | traced  | new sp  |         |
 * |------------+---------+---------+---------+---------|
 * | SYS_XIT    | parent  | exitee  |         |         |
 * |------------+---------+---------+---------+---------|
 * | SYS_GETSP  | proc nr |         |         |         |
 * |------------+---------+---------+---------+---------|
 * | SYS_TIMES  | proc nr |         | buf ptr |         |
 * |------------+---------+---------+---------+---------|
 * | SYS_ABORT  |         |         |         |         |
 * |------------+---------+---------+---------+---------|
 * | SYS_FRESH  | proc nr | data_cl |         |         |
 * |------------+---------+---------+---------+---------|
 * | SYS_GBOOT  | proc nr |         |         | bootptr |
 * |------------+---------+---------+---------+---------|
 * | SYS_GETMAP | proc nr |         |         | map ptr |
 * ------------------------------------------------------
 *
 *    m_type          m1_i1     m1_i2     m1_i3       m1_p1
 * ----------------+---------+---------+---------+--------------
 * | SYS_VCOPY     |  src p  |  dst p  | vec siz | vc addr     |
 * |---------------+---------+---------+---------+-------------|
 * | SYS_SENDSIG   | proc nr |         |         | smp         |
 * |---------------+---------+---------+---------+-------------|
 * | SYS_SIGRETURN | proc nr |         |         | scp         |
 * |---------------+---------+---------+---------+-------------|
 * | SYS_ENDSIG    | proc nr |         |         |             |
 * -------------------------------------------------------------
 *
 *    m_type       m2_i1     m2_i2     m2_l1     m2_l2
 * ------------------------------------------------------
 * | SYS_TRACE  | proc_nr | request |  addr   |  data   |
 * ------------------------------------------------------
 *
 *
 *    m_type       m6_i1     m6_i2     m6_i3     m6_f1
 * ------------------------------------------------------
 * | SYS_KILL   | proc_nr  |  sig    |         |         |
 * ------------------------------------------------------
 *
 *
 *    m_type      m5_c1   m5_i1    m5_l1   m5_c2   m5_i2    m5_l2   m5_l3
 * --------------------------------------------------------------------------
 * | SYS_COPY   |src seg|src proc|src vir|dst seg|dst proc|dst vir| byte ct |
 * --------------------------------------------------------------------------
 * | SYS_UMAP   |  seg  |proc nr |vir adr|       |        |       | byte ct |
 * --------------------------------------------------------------------------
 *
 *
 *    m_type      m1_i1      m1_i2      m1_i3
 * |------------+----------+----------+----------
 * | SYS_MEM    | mem base | mem size | tot mem |
 * ----------------------------------------------
 *
 *
 *    m_type        m1_i1      m1_i2      m1_i3
 * ---------------+----------+----------+----------
 * | SYS_BLOCK    | proc_nr  |		    |		  |
 * |----------------------------------------------|
 * | SYS_UNBLOCK  | proc_nr  |		    |		  |
 * ------------------------------------------------
 *
 *
 * In addition to the main sys_task() entry point, there are 5 other minor
 * entry points:
 *   cause_sig:	take action to cause a signal to occur, sooner or later
 *   inform:	tell MM about pending signals
 *   numap:	umap D segment starting from process number instead of pointer
 *   umap:	compute the physical address for a given virtual address
 *   alloc_segments: allocate segments for 8088 or higher processor
 */

#include "kernel.h"
#include <signal.h>
#include <unistd.h>
#include <sys/sigcontext.h>
#include <sys/ptrace.h>
#include <minix/boot.h>
#include <minix/callnr.h>
#include <minix/com.h>
#include "proc.h"
#if (CHIP == INTEL)
#include "protect.h"
#endif

/* PSW masks. */
#define IF_MASK 0x00000200
#define IOPL_MASK 0x003000

PRIVATE message m;

FORWARD _PROTOTYPE( int do_abort, (message *m_ptr) );
FORWARD _PROTOTYPE( int do_copy, (message *m_ptr) );
FORWARD _PROTOTYPE( int do_exec, (message *m_ptr) );
FORWARD _PROTOTYPE( int do_fork, (message *m_ptr) );
FORWARD _PROTOTYPE( int do_gboot, (message *m_ptr) );
FORWARD _PROTOTYPE( int do_getsp, (message *m_ptr) );
FORWARD _PROTOTYPE( int do_kill, (message *m_ptr) );
FORWARD _PROTOTYPE( int do_mem, (message *m_ptr) );
FORWARD _PROTOTYPE( int do_newmap, (message *m_ptr) );
FORWARD _PROTOTYPE( int do_sendsig, (message *m_ptr) );
FORWARD _PROTOTYPE( int do_sigreturn, (message *m_ptr) );
FORWARD _PROTOTYPE( int do_endsig, (message *m_ptr) );
FORWARD _PROTOTYPE( int do_times, (message *m_ptr) );
FORWARD _PROTOTYPE( int do_trace, (message *m_ptr) );
FORWARD _PROTOTYPE( int do_umap, (message *m_ptr) );
FORWARD _PROTOTYPE( int do_xit, (message *m_ptr) );
FORWARD _PROTOTYPE( int do_vcopy, (message *m_ptr) );
FORWARD _PROTOTYPE( int do_getmap, (message *m_ptr) );

FORWARD _PROTOTYPE( int do_block, (message *m_ptr) );
FORWARD _PROTOTYPE( int do_unblock, (message *m_ptr) );

#if (SHADOWING == 1)
FORWARD _PROTOTYPE( int do_fresh, (message *m_ptr) );
#endif

/*===========================================================================*
 *				sys_task				     *
 *===========================================================================*/
PUBLIC void sys_task()
{
/* Main entry point of sys_task.  Get the message and dispatch on type. */

  register int r;

  while (TRUE) {
	receive(ANY, &m);

	switch (m.m_type) {	/* which system call */
	    case SYS_FORK:	r = do_fork(&m);	break;
	    case SYS_NEWMAP:	r = do_newmap(&m);	break;
	    case SYS_GETMAP:	r = do_getmap(&m);	break;
	    case SYS_EXEC:	r = do_exec(&m);	break;
	    case SYS_XIT:	r = do_xit(&m);		break;
	    case SYS_GETSP:	r = do_getsp(&m);	break;
	    case SYS_TIMES:	r = do_times(&m);	break;
	    case SYS_ABORT:	r = do_abort(&m);	break;
#if (SHADOWING == 1)
	    case SYS_FRESH:	r = do_fresh(&m);	break;
#endif
	    case SYS_SENDSIG:	r = do_sendsig(&m);	break;
	    case SYS_SIGRETURN: r = do_sigreturn(&m);	break;
	    case SYS_KILL:	r = do_kill(&m);	break;
	    case SYS_ENDSIG:	r = do_endsig(&m);	break;
	    case SYS_COPY:	r = do_copy(&m);	break;
            case SYS_VCOPY:	r = do_vcopy(&m);	break;
	    case SYS_GBOOT:	r = do_gboot(&m);	break;
	    case SYS_MEM:	r = do_mem(&m);		break;
	    case SYS_UMAP:	r = do_umap(&m);	break;
	    case SYS_TRACE:	r = do_trace(&m);	break;
	    case SYS_BLOCK:	r = do_block(&m);	break;
	    case SYS_UNBLOCK:	r = do_unblock(&m);	break;
	    default:		r = E_BAD_FCN;
	}

	m.m_type = r;		/* 'r' reports status of call */
	send(m.m_source, &m);	/* send reply to caller */
  }
}

/*===========================================================================*
 *				do_block					     *
 *===========================================================================*/
 PRIVATE int do_block(m_ptr)
 register message *m_ptr;	/* pointer to request message */
{
/* Handle sys_block().  A process is blocked by semaphore. */

	register struct proc *rp;

	rp = proc_addr(m_ptr->PROC1);

	if (rp->p_flags == 0) lock_unready(rp);

	rp->p_flags |= BLOCK_X_SEM; /* bloquea x semaforo */

	return(OK);
}

/*===========================================================================*
 *				do_unblock					     *
 *===========================================================================*/
  PRIVATE int do_unblock(m_ptr)
  register message *m_ptr;	/* pointer to request message */
 {
/* Handle sys_unblock().  Process is unblocked by semaphore. */

 	register struct proc *rp;

 	rp = proc_addr(m_ptr->PROC1); /* process to unblocks id */

 	rp->p_flags &= ~BLOCK_X_SEM; /* desbloquea x semaforo */

 	if (rp->p_flags == 0) lock_ready(rp);

 	return(OK);
}

/*===========================================================================*
 *				do_fork					     *
 *===========================================================================*/
PRIVATE int do_fork(m_ptr)
register message *m_ptr;	/* pointer to request message */
{
/* Handle sys_fork().  m_ptr->PROC1 has forked.  The child is m_ptr->PROC2. */

#if (CHIP == INTEL)
  reg_t old_ldt_sel;
#endif
  register struct proc *rpc;
  struct proc *rpp;

  if (!isoksusern(m_ptr->PROC1) || !isoksusern(m_ptr->PROC2))
	return(E_BAD_PROC);
  rpp = proc_addr(m_ptr->PROC1);
  rpc = proc_addr(m_ptr->PROC2);

  /* Copy parent 'proc' struct to child. */
#if (CHIP == INTEL)
  old_ldt_sel = rpc->p_ldt_sel;	/* stop this being obliterated by copy */
#endif

  *rpc = *rpp;			/* copy 'proc' struct */

#if (CHIP == INTEL)
  rpc->p_ldt_sel = old_ldt_sel;
#endif
  rpc->p_nr = m_ptr->PROC2;	/* this was obliterated by copy */

#if (SHADOWING == 0)
  rpc->p_flags |= NO_MAP;	/* inhibit the process from running */
#endif

  rpc->p_flags &= ~(PENDING | SIG_PENDING | P_STOP);

  /* Only 1 in group should have PENDING, child does not inherit trace status*/
  sigemptyset(&rpc->p_pending);
  rpc->p_pendcount = 0;
  rpc->p_pid = m_ptr->PID;	/* install child's pid */
  rpc->p_reg.retreg = 0;	/* child sees pid = 0 to know it is child */

  rpc->user_time = 0;		/* set all the accounting times to 0 */
  rpc->sys_time = 0;
  rpc->child_utime = 0;
  rpc->child_stime = 0;

#if (SHADOWING == 1)
  rpc->p_nflips = 0;
  mkshadow(rpp, (phys_clicks)m_ptr->m1_p1);	/* run child first */
#endif

  return(OK);
}


/*===========================================================================*
 *				do_newmap				     *
 *===========================================================================*/
PRIVATE int do_newmap(m_ptr)
message *m_ptr;			/* pointer to request message */
{
/* Handle sys_newmap().  Fetch the memory map from MM. */

  register struct proc *rp;
  phys_bytes src_phys;
  int caller;			/* whose space has the new map (usually MM) */
  int k;			/* process whose map is to be loaded */
  int old_flags;		/* value of flags before modification */
  struct mem_map *map_ptr;	/* virtual address of map inside caller (MM) */

  /* Extract message parameters and copy new memory map from MM. */
  caller = m_ptr->m_source;
  k = m_ptr->PROC1;
  map_ptr = (struct mem_map *) m_ptr->MEM_PTR;
  if (!isokprocn(k)) return(E_BAD_PROC);
  rp = proc_addr(k);		/* ptr to entry of user getting new map */

  /* Copy the map from MM. */
  src_phys = umap(proc_addr(caller), D, (vir_bytes) map_ptr, sizeof(rp->p_map));
  if (src_phys == 0) panic("bad call to sys_newmap", NO_NUM);
  phys_copy(src_phys, vir2phys(rp->p_map), (phys_bytes) sizeof(rp->p_map));

#if (SHADOWING == 0)
#if (CHIP != M68000)
  alloc_segments(rp);
#else
  pmmu_init_proc(rp);
#endif
  old_flags = rp->p_flags;	/* save the previous value of the flags */
  rp->p_flags &= ~NO_MAP;
  if (old_flags != 0 && rp->p_flags == 0) lock_ready(rp);
#endif

  return(OK);
}


/*===========================================================================*
 *				do_getmap				     *
 *===========================================================================*/
PRIVATE int do_getmap(m_ptr)
message *m_ptr;			/* pointer to request message */
{
/* Handle sys_getmap().  Report the memory map to MM. */

  register struct proc *rp;
  phys_bytes dst_phys;
  int caller;			/* where the map has to be stored */
  int k;			/* process whose map is to be loaded */
  struct mem_map *map_ptr;	/* virtual address of map inside caller (MM) */

  /* Extract message parameters and copy new memory map to MM. */
  caller = m_ptr->m_source;
  k = m_ptr->PROC1;
  map_ptr = (struct mem_map *) m_ptr->MEM_PTR;

  if (!isokprocn(k))
	panic("do_getmap got bad proc: ", m_ptr->PROC1);

  rp = proc_addr(k);		/* ptr to entry of the map */

  /* Copy the map to MM. */
  dst_phys = umap(proc_addr(caller), D, (vir_bytes) map_ptr, sizeof(rp->p_map));
  if (dst_phys == 0) panic("bad call to sys_getmap", NO_NUM);
  phys_copy(vir2phys(rp->p_map), dst_phys, sizeof(rp->p_map));

  return(OK);
}


/*===========================================================================*
 *				do_exec					     *
 *===========================================================================*/
PRIVATE int do_exec(m_ptr)
register message *m_ptr;	/* pointer to request message */
{
/* Handle sys_exec().  A process has done a successful EXEC. Patch it up. */

  register struct proc *rp;
  reg_t sp;			/* new sp */
  phys_bytes phys_name;
  char *np;
#define NLEN (sizeof(rp->p_name)-1)

  if (!isoksusern(m_ptr->PROC1)) return E_BAD_PROC;
  /* PROC2 field is used as flag to indicate process is being traced */
  if (m_ptr->PROC2) cause_sig(m_ptr->PROC1, SIGTRAP);
  sp = (reg_t) m_ptr->STACK_PTR;
  rp = proc_addr(m_ptr->PROC1);
  rp->p_reg.sp = sp;		/* set the stack pointer */
#if (CHIP == M68000)
  rp->p_splow = sp;		/* set the stack pointer low water */
#ifdef FPP
  /* Initialize fpp for this process */
  fpp_new_state(rp);
#endif
#endif
  rp->p_reg.pc = (reg_t) m_ptr->IP_PTR;	/* set pc */
  rp->p_alarm = 0;		/* reset alarm timer */
  rp->p_flags &= ~RECEIVING;	/* MM does not reply to EXEC call */
  if (rp->p_flags == 0) lock_ready(rp);

  /* Save command name for debugging, ps(1) output, etc. */
  phys_name = numap(m_ptr->m_source, (vir_bytes) m_ptr->NAME_PTR,
							(vir_bytes) NLEN);
  if (phys_name != 0) {
	phys_copy(phys_name, vir2phys(rp->p_name), (phys_bytes) NLEN);
	for (np = rp->p_name; (*np & BYTE) >= ' '; np++) {}
	*np = 0;
  }
  return(OK);
}


/*===========================================================================*
 *				do_xit					     *
 *===========================================================================*/
PRIVATE int do_xit(m_ptr)
message *m_ptr;			/* pointer to request message */
{
/* Handle sys_xit().  A process has exited. */

  register struct proc *rp, *rc;
  struct proc *np, *xp;
  int parent;			/* number of exiting proc's parent */
  int proc_nr;			/* number of process doing the exit */
  phys_clicks base, size;

  parent = m_ptr->PROC1;	/* slot number of parent process */
  proc_nr = m_ptr->PROC2;	/* slot number of exiting process */
  if (!isoksusern(parent) || !isoksusern(proc_nr)) return(E_BAD_PROC);
  rp = proc_addr(parent);
  rc = proc_addr(proc_nr);
  lock();
  rp->child_utime += rc->user_time + rc->child_utime;	/* accum child times */
  rp->child_stime += rc->sys_time + rc->child_stime;
  unlock();
  rc->p_alarm = 0;		/* turn off alarm timer */
  if (rc->p_flags == 0) lock_unready(rc);

#if (SHADOWING == 1)
  rmshadow(rc, &base, &size);
  m_ptr->m1_i1 = (int)base;
  m_ptr->m1_i2 = (int)size;
#endif

  strcpy(rc->p_name, "<noname>");	/* process no longer has a name */

  /* If the process being terminated happens to be queued trying to send a
   * message (i.e., the process was killed by a signal, rather than it doing an
   * EXIT), then it must be removed from the message queues.
   */
  if (rc->p_flags & SENDING) {
	/* Check all proc slots to see if the exiting process is queued. */
	for (rp = BEG_PROC_ADDR; rp < END_PROC_ADDR; rp++) {
		if (rp->p_callerq == NIL_PROC) continue;
		if (rp->p_callerq == rc) {
			/* Exiting process is on front of this queue. */
			rp->p_callerq = rc->p_sendlink;
			break;
		} else {
			/* See if exiting process is in middle of queue. */
			np = rp->p_callerq;
			while ( ( xp = np->p_sendlink) != NIL_PROC)
				if (xp == rc) {
					np->p_sendlink = xp->p_sendlink;
					break;
				} else {
					np = xp;
				}
		}
	}
  }
#if (CHIP == M68000) && (SHADOWING == 0)
  pmmu_delete(rc);	/* we're done remove tables */
#endif

  if (rc->p_flags & PENDING) --sig_procs;
  sigemptyset(&rc->p_pending);
  rc->p_pendcount = 0;
  rc->p_flags = P_SLOT_FREE;
  return(OK);
}


/*===========================================================================*
 *				do_getsp				     *
 *===========================================================================*/
PRIVATE int do_getsp(m_ptr)
register message *m_ptr;	/* pointer to request message */
{
/* Handle sys_getsp().  MM wants to know what sp is. */

  register struct proc *rp;

  if (!isoksusern(m_ptr->PROC1)) return(E_BAD_PROC);
  rp = proc_addr(m_ptr->PROC1);
  m_ptr->STACK_PTR = (char *) rp->p_reg.sp;	/* return sp here (bad type) */
  return(OK);
}


/*===========================================================================*
 *				do_times				     *
 *===========================================================================*/
PRIVATE int do_times(m_ptr)
register message *m_ptr;	/* pointer to request message */
{
/* Handle sys_times().  Retrieve the accounting information. */

  register struct proc *rp;

  if (!isoksusern(m_ptr->PROC1)) return E_BAD_PROC;
  rp = proc_addr(m_ptr->PROC1);

  /* Insert the times needed by the TIMES system call in the message. */
  lock();			/* halt the volatile time counters in rp */
  m_ptr->USER_TIME   = rp->user_time;
  m_ptr->SYSTEM_TIME = rp->sys_time;
  unlock();
  m_ptr->CHILD_UTIME = rp->child_utime;
  m_ptr->CHILD_STIME = rp->child_stime;
  m_ptr->BOOT_TICKS  = get_uptime();
  return(OK);
}


/*===========================================================================*
 *				do_abort				     *
 *===========================================================================*/
PRIVATE int do_abort(m_ptr)
message *m_ptr;			/* pointer to request message */
{
/* Handle sys_abort.  MINIX is unable to continue.  Terminate operation. */
  char monitor_code[64];
  phys_bytes src_phys;

  if (m_ptr->m1_i1 == RBT_MONITOR) {
	/* The monitor is to run user specified instructions. */
	src_phys = numap(m_ptr->m_source, (vir_bytes) m_ptr->m1_p1,
					(vir_bytes) sizeof(monitor_code));
	if (src_phys == 0) panic("bad monitor code from", m_ptr->m_source);
	phys_copy(src_phys, vir2phys(monitor_code),
					(phys_bytes) sizeof(monitor_code));
	reboot_code = vir2phys(monitor_code);
  }
  wreboot(m_ptr->m1_i1);
  return(OK);			/* pro-forma (really EDISASTER) */
}


#if (SHADOWING == 1)
/*===========================================================================*
 *				do_fresh				     *
 *===========================================================================*/
PRIVATE int do_fresh(m_ptr)     /* for 68000 only */
message *m_ptr;			/* pointer to request message */
{
/* Handle sys_fresh.  Start with fresh process image during EXEC. */

  register struct proc *p;
  int proc_nr;			/* number of process doing the exec */
  phys_clicks base, size;
  phys_clicks c1, nc;

  proc_nr = m_ptr->PROC1;	/* slot number of exec-ing process */
  if (!isokprocn(proc_nr)) return(E_BAD_PROC);
  p = proc_addr(proc_nr);
  rmshadow(p, &base, &size);
  do_newmap(m_ptr);
  c1 = p->p_map[D].mem_phys;
  nc = p->p_map[S].mem_phys - p->p_map[D].mem_phys + p->p_map[S].mem_len;
  c1 += m_ptr->m1_i2;
  nc -= m_ptr->m1_i2;
  zeroclicks(c1, nc);
  m_ptr->m1_i1 = (int)base;
  m_ptr->m1_i2 = (int)size;
  return(OK);
}
#endif /* (SHADOWING == 1) */


/*===========================================================================*
 *			      do_sendsig				     *
 *===========================================================================*/
PRIVATE int do_sendsig(m_ptr)
message *m_ptr;			/* pointer to request message */
{
/* Handle sys_sendsig, POSIX-style signal */

  struct sigmsg smsg;
  register struct proc *rp;
  phys_bytes src_phys, dst_phys;
  struct sigcontext sc, *scp;
  struct sigframe fr, *frp;

  if (!isokusern(m_ptr->PROC1)) return(E_BAD_PROC);
  rp = proc_addr(m_ptr->PROC1);

  /* Get the sigmsg structure into our address space.  */
  src_phys = umap(proc_addr(MM_PROC_NR), D, (vir_bytes) m_ptr->SIG_CTXT_PTR,
		  (vir_bytes) sizeof(struct sigmsg));
  if (src_phys == 0)
	panic("do_sendsig can't signal: bad sigmsg address from MM", NO_NUM);
  phys_copy(src_phys, vir2phys(&smsg), (phys_bytes) sizeof(struct sigmsg));

  /* Compute the usr stack pointer value where sigcontext will be stored. */
  scp = (struct sigcontext *) smsg.sm_stkptr - 1;

  /* Copy the registers to the sigcontext structure. */
  memcpy(&sc.sc_regs, &rp->p_reg, sizeof(struct sigregs));

  /* Finish the sigcontext initialization. */
  sc.sc_flags = SC_SIGCONTEXT;

  sc.sc_mask = smsg.sm_mask;

  /* Copy the sigcontext structure to the user's stack. */
  dst_phys = umap(rp, D, (vir_bytes) scp,
		  (vir_bytes) sizeof(struct sigcontext));
  if (dst_phys == 0) return(EFAULT);
  phys_copy(vir2phys(&sc), dst_phys, (phys_bytes) sizeof(struct sigcontext));

  /* Initialize the sigframe structure. */
  frp = (struct sigframe *) scp - 1;
  fr.sf_scpcopy = scp;
  fr.sf_retadr2= (void (*)()) rp->p_reg.pc;
  fr.sf_fp = rp->p_reg.fp;
  rp->p_reg.fp = (reg_t) &frp->sf_fp;
  fr.sf_scp = scp;
  fr.sf_code = 0;	/* XXX - should be used for type of FP exception */
  fr.sf_signo = smsg.sm_signo;
  fr.sf_retadr = (void (*)()) smsg.sm_sigreturn;

  /* Copy the sigframe structure to the user's stack. */
  dst_phys = umap(rp, D, (vir_bytes) frp, (vir_bytes) sizeof(struct sigframe));
  if (dst_phys == 0) return(EFAULT);
  phys_copy(vir2phys(&fr), dst_phys, (phys_bytes) sizeof(struct sigframe));

  /* Reset user registers to execute the signal handler. */
  rp->p_reg.sp = (reg_t) frp;
  rp->p_reg.pc = (reg_t) smsg.sm_sighandler;

  return(OK);
}

/*===========================================================================*
 *			      do_sigreturn				     *
 *===========================================================================*/
PRIVATE int do_sigreturn(m_ptr)
register message *m_ptr;
{
/* POSIX style signals require sys_sigreturn to put things in order before the
 * signalled process can resume execution
 */

  struct sigcontext sc;
  register struct proc *rp;
  phys_bytes src_phys;

  if (!isokusern(m_ptr->PROC1)) return(E_BAD_PROC);
  rp = proc_addr(m_ptr->PROC1);

  /* Copy in the sigcontext structure. */
  src_phys = umap(rp, D, (vir_bytes) m_ptr->SIG_CTXT_PTR,
		  (vir_bytes) sizeof(struct sigcontext));
  if (src_phys == 0) return(EFAULT);
  phys_copy(src_phys, vir2phys(&sc), (phys_bytes) sizeof(struct sigcontext));

  /* Make sure that this is not just a jmp_buf. */
  if ((sc.sc_flags & SC_SIGCONTEXT) == 0) return(EINVAL);

  /* Fix up only certain key registers if the compiler doesn't use
   * register variables within functions containing setjmp.
   */
  if (sc.sc_flags & SC_NOREGLOCALS) {
	rp->p_reg.retreg = sc.sc_retreg;
	rp->p_reg.fp = sc.sc_fp;
	rp->p_reg.pc = sc.sc_pc;
	rp->p_reg.sp = sc.sc_sp;
	return (OK);
  }
  sc.sc_psw  = rp->p_reg.psw;

#if (CHIP == INTEL)
  /* Don't panic kernel if user gave bad selectors. */
  sc.sc_cs = rp->p_reg.cs;
  sc.sc_ds = rp->p_reg.ds;
  sc.sc_es = rp->p_reg.es;
#if _WORD_SIZE == 4
  sc.sc_fs = rp->p_reg.fs;
  sc.sc_gs = rp->p_reg.gs;
#endif
#endif

  /* Restore the registers. */
  memcpy(&rp->p_reg, (char *)&sc.sc_regs, sizeof(struct sigregs));

  return(OK);
}

/*===========================================================================*
 *				do_kill					     *
 *===========================================================================*/
PRIVATE int do_kill(m_ptr)
register message *m_ptr;	/* pointer to request message */
{
/* Handle sys_kill(). Cause a signal to be sent to a process via MM.
 * Note that this has nothing to do with the kill (2) system call, this
 * is how the FS (and possibly other servers) get access to cause_sig to
 * send a KSIG message to MM
 */

  if (!isokusern(m_ptr->PR)) return(E_BAD_PROC);
  cause_sig(m_ptr->PR, m_ptr->SIGNUM);
  return(OK);
}


/*===========================================================================*
 *			      do_endsig					     *
 *===========================================================================*/
PRIVATE int do_endsig(m_ptr)
register message *m_ptr;	/* pointer to request message */
{
/* Finish up after a KSIG-type signal, caused by a SYS_KILL message or a call
 * to cause_sig by a task
 */

  register struct proc *rp;

  if (!isokusern(m_ptr->PROC1)) return(E_BAD_PROC);
  rp = proc_addr(m_ptr->PROC1);

  /* MM has finished one KSIG. */
  if (rp->p_pendcount != 0 && --rp->p_pendcount == 0
      && (rp->p_flags &= ~SIG_PENDING) == 0)
	lock_ready(rp);
  return(OK);
}

/*===========================================================================*
 *				do_copy					     *
 *===========================================================================*/
PRIVATE int do_copy(m_ptr)
register message *m_ptr;	/* pointer to request message */
{
/* Handle sys_copy().  Copy data for MM or FS. */

  int src_proc, dst_proc, src_space, dst_space;
  vir_bytes src_vir, dst_vir;
  phys_bytes src_phys, dst_phys, bytes;

  /* Dismember the command message. */
  src_proc = m_ptr->SRC_PROC_NR;
  dst_proc = m_ptr->DST_PROC_NR;
  src_space = m_ptr->SRC_SPACE;
  dst_space = m_ptr->DST_SPACE;
  src_vir = (vir_bytes) m_ptr->SRC_BUFFER;
  dst_vir = (vir_bytes) m_ptr->DST_BUFFER;
  bytes = (phys_bytes) m_ptr->COPY_BYTES;

  /* Compute the source and destination addresses and do the copy. */
#if (SHADOWING == 0)
  if (src_proc == ABS)
	src_phys = (phys_bytes) m_ptr->SRC_BUFFER;
  else {
	if (bytes != (vir_bytes) bytes)
		/* This would happen for 64K segments and 16-bit vir_bytes.
		 * It would happen a lot for do_fork except MM uses ABS
		 * copies for that case.
		 */
		panic("overflow in count in do_copy", NO_NUM);
#endif

	src_phys = umap(proc_addr(src_proc), src_space, src_vir,
			(vir_bytes) bytes);
#if (SHADOWING == 0)
	}
#endif

#if (SHADOWING == 0)
  if (dst_proc == ABS)
	dst_phys = (phys_bytes) m_ptr->DST_BUFFER;
  else
#endif
	dst_phys = umap(proc_addr(dst_proc), dst_space, dst_vir,
			(vir_bytes) bytes);

  if (src_phys == 0 || dst_phys == 0) return(EFAULT);
  phys_copy(src_phys, dst_phys, bytes);
  return(OK);
}


/*===========================================================================*
 *				do_vcopy				     *
 *===========================================================================*/
PRIVATE int do_vcopy(m_ptr)
register message *m_ptr;	/* pointer to request message */
{
/* Handle sys_vcopy(). Copy multiple blocks of memory */

  int src_proc, dst_proc, vect_s, i;
  vir_bytes src_vir, dst_vir, vect_addr;
  phys_bytes src_phys, dst_phys, bytes;
  cpvec_t cpvec_table[CPVEC_NR];

  /* Dismember the command message. */
  src_proc = m_ptr->m1_i1;
  dst_proc = m_ptr->m1_i2;
  vect_s = m_ptr->m1_i3;
  vect_addr = (vir_bytes)m_ptr->m1_p1;

  if (vect_s > CPVEC_NR) return EDOM;

  src_phys= numap (m_ptr->m_source, vect_addr, vect_s * sizeof(cpvec_t));
  if (!src_phys) return EFAULT;
  phys_copy(src_phys, vir2phys(cpvec_table),
				(phys_bytes) (vect_s * sizeof(cpvec_t)));

  for (i = 0; i < vect_s; i++) {
	src_vir= cpvec_table[i].cpv_src;
	dst_vir= cpvec_table[i].cpv_dst;
	bytes= cpvec_table[i].cpv_size;
	src_phys = numap(src_proc,src_vir,(vir_bytes)bytes);
	dst_phys = numap(dst_proc,dst_vir,(vir_bytes)bytes);
	if (src_phys == 0 || dst_phys == 0) return(EFAULT);
	phys_copy(src_phys, dst_phys, bytes);
  }
  return(OK);
}


/*==========================================================================*
 *				do_gboot				    *
 *==========================================================================*/
PUBLIC struct bparam_s boot_parameters;

PRIVATE int do_gboot(m_ptr)
message *m_ptr;			/* pointer to request message */
{
/* Copy the boot parameters.  Normally only called during fs init. */

  phys_bytes dst_phys;

  dst_phys = umap(proc_addr(m_ptr->PROC1), D, (vir_bytes) m_ptr->MEM_PTR,
				(vir_bytes) sizeof(boot_parameters));
  if (dst_phys == 0) panic("bad call to SYS_GBOOT", NO_NUM);
  phys_copy(vir2phys(&boot_parameters), dst_phys,
				(phys_bytes) sizeof(boot_parameters));
  return(OK);
}


/*===========================================================================*
 *				do_mem					     *
 *===========================================================================*/
PRIVATE int do_mem(m_ptr)
register message *m_ptr;	/* pointer to request message */
{
/* Return the base and size of the next chunk of memory. */

  struct memory *memp;

  for (memp = mem; memp < &mem[NR_MEMS]; ++memp) {
	m_ptr->m1_i1 = memp->base;
	m_ptr->m1_i2 = memp->size;
	m_ptr->m1_i3 = tot_mem_size;
	memp->size = 0;
	if (m_ptr->m1_i2 != 0) break;		/* found a chunk */
  }
  return(OK);
}


/*==========================================================================*
 *				do_umap					    *
 *==========================================================================*/
PRIVATE int do_umap(m_ptr)
register message *m_ptr;	/* pointer to request message */
{
/* Same as umap(), for non-kernel processes. */

  m_ptr->SRC_BUFFER = umap(proc_addr((int) m_ptr->SRC_PROC_NR),
                           (int) m_ptr->SRC_SPACE,
                           (vir_bytes) m_ptr->SRC_BUFFER,
                           (vir_bytes) m_ptr->COPY_BYTES);
  return(OK);
}


/*==========================================================================*
 *				do_trace				    *
 *==========================================================================*/
#define TR_PROCNR	(m_ptr->m2_i1)
#define TR_REQUEST	(m_ptr->m2_i2)
#define TR_ADDR		((vir_bytes) m_ptr->m2_l1)
#define TR_DATA		(m_ptr->m2_l2)
#define TR_VLSIZE	((vir_bytes) sizeof(long))

PRIVATE int do_trace(m_ptr)
register message *m_ptr;
{
/* Handle the debugging commands supported by the ptrace system call
 * The commands are:
 * T_STOP	stop the process
 * T_OK		enable tracing by parent for this process
 * T_GETINS	return value from instruction space
 * T_GETDATA	return value from data space
 * T_GETUSER	return value from user process table
 * T_SETINS	set value from instruction space
 * T_SETDATA	set value from data space
 * T_SETUSER	set value in user process table
 * T_RESUME	resume execution
 * T_EXIT	exit
 * T_STEP	set trace bit
 *
 * The T_OK and T_EXIT commands are handled completely by the memory manager,
 * all others come here.
 */

  register struct proc *rp;
  phys_bytes src, dst;
  int i;

  rp = proc_addr(TR_PROCNR);
  if (rp->p_flags & P_SLOT_FREE) return(EIO);
  switch (TR_REQUEST) {
  case T_STOP:			/* stop process */
	if (rp->p_flags == 0) lock_unready(rp);
	rp->p_flags |= P_STOP;
	rp->p_reg.psw &= ~TRACEBIT;	/* clear trace bit */
	return(OK);

  case T_GETINS:		/* return value from instruction space */
	if (rp->p_map[T].mem_len != 0) {
		if ((src = umap(rp, T, TR_ADDR, TR_VLSIZE)) == 0) return(EIO);
		phys_copy(src, vir2phys(&TR_DATA), (phys_bytes) sizeof(long));
		break;
	}
	/* Text space is actually data space - fall through. */

  case T_GETDATA:		/* return value from data space */
	if ((src = umap(rp, D, TR_ADDR, TR_VLSIZE)) == 0) return(EIO);
	phys_copy(src, vir2phys(&TR_DATA), (phys_bytes) sizeof(long));
	break;

  case T_GETUSER:		/* return value from process table */
	if ((TR_ADDR & (sizeof(long) - 1)) != 0 ||
	    TR_ADDR > sizeof(struct proc) - sizeof(long))
		return(EIO);
	TR_DATA = *(long *) ((char *) rp + (int) TR_ADDR);
	break;

  case T_SETINS:		/* set value in instruction space */
	if (rp->p_map[T].mem_len != 0) {
		if ((dst = umap(rp, T, TR_ADDR, TR_VLSIZE)) == 0) return(EIO);
		phys_copy(vir2phys(&TR_DATA), dst, (phys_bytes) sizeof(long));
		TR_DATA = 0;
		break;
	}
	/* Text space is actually data space - fall through. */

  case T_SETDATA:			/* set value in data space */
	if ((dst = umap(rp, D, TR_ADDR, TR_VLSIZE)) == 0) return(EIO);
	phys_copy(vir2phys(&TR_DATA), dst, (phys_bytes) sizeof(long));
	TR_DATA = 0;
	break;

  case T_SETUSER:			/* set value in process table */
	if ((TR_ADDR & (sizeof(reg_t) - 1)) != 0 ||
	     TR_ADDR > sizeof(struct stackframe_s) - sizeof(reg_t))
		return(EIO);
	i = (int) TR_ADDR;
#if (CHIP == INTEL)
	/* Altering segment registers might crash the kernel when it
	 * tries to load them prior to restarting a process, so do
	 * not allow it.
	 */
	if (i == (int) &((struct proc *) 0)->p_reg.cs ||
	    i == (int) &((struct proc *) 0)->p_reg.ds ||
	    i == (int) &((struct proc *) 0)->p_reg.es ||
#if _WORD_SIZE == 4
	    i == (int) &((struct proc *) 0)->p_reg.gs ||
	    i == (int) &((struct proc *) 0)->p_reg.fs ||
#endif
	    i == (int) &((struct proc *) 0)->p_reg.ss)
		return(EIO);
#endif
	if (i == (int) &((struct proc *) 0)->p_reg.psw)
		/* only selected bits are changeable */
		SETPSW(rp, TR_DATA);
	else
		*(reg_t *) ((char *) &rp->p_reg + i) = (reg_t) TR_DATA;
	TR_DATA = 0;
	break;

  case T_RESUME:		/* resume execution */
	rp->p_flags &= ~P_STOP;
	if (rp->p_flags == 0) lock_ready(rp);
	TR_DATA = 0;
	break;

  case T_STEP:			/* set trace bit */
	rp->p_reg.psw |= TRACEBIT;
	rp->p_flags &= ~P_STOP;
	if (rp->p_flags == 0) lock_ready(rp);
	TR_DATA = 0;
	break;

  default:
	return(EIO);
  }
  return(OK);
}

/*===========================================================================*
 *				cause_sig				     *
 *===========================================================================*/
PUBLIC void cause_sig(proc_nr, sig_nr)
int proc_nr;			/* process to be signalled */
int sig_nr;			/* signal to be sent, 1 to _NSIG */
{
/* A task wants to send a signal to a process.   Examples of such tasks are:
 *   TTY wanting to cause SIGINT upon getting a DEL
 *   CLOCK wanting to cause SIGALRM when timer expires
 * FS also uses this to send a signal, via the SYS_KILL message.
 * Signals are handled by sending a message to MM.  The tasks don't dare do
 * that directly, for fear of what would happen if MM were busy.  Instead they
 * call cause_sig, which sets bits in p_pending, and then carefully checks to
 * see if MM is free.  If so, a message is sent to it.  If not, when it becomes
 * free, a message is sent.  The process being signaled is blocked while MM
 * has not seen or finished with all signals for it.  These signals are
 * counted in p_pendcount, and the SIG_PENDING flag is kept nonzero while
 * there are some.  It is not sufficient to ready the process when MM is
 * informed, because MM can block waiting for FS to do a core dump.
 */

  register struct proc *rp, *mmp;

  rp = proc_addr(proc_nr);
  if (sigismember(&rp->p_pending, sig_nr))
	return;			/* this signal already pending */
  sigaddset(&rp->p_pending, sig_nr);
  ++rp->p_pendcount;		/* count new signal pending */
  if (rp->p_flags & PENDING)
	return;			/* another signal already pending */
  if (rp->p_flags == 0) lock_unready(rp);
  rp->p_flags |= PENDING | SIG_PENDING;
  ++sig_procs;			/* count new process pending */

  mmp = proc_addr(MM_PROC_NR);
  if ( ((mmp->p_flags & RECEIVING) == 0) || mmp->p_getfrom != ANY) return;
  inform();
}


/*===========================================================================*
 *				inform					     *
 *===========================================================================*/
PUBLIC void inform()
{
/* When a signal is detected by the kernel (e.g., DEL), or generated by a task
 * (e.g. clock task for SIGALRM), cause_sig() is called to set a bit in the
 * p_pending field of the process to signal.  Then inform() is called to see
 * if MM is idle and can be told about it.  Whenever MM blocks, a check is
 * made to see if 'sig_procs' is nonzero; if so, inform() is called.
 */

  register struct proc *rp;

  /* MM is waiting for new input.  Find a process with pending signals. */
  for (rp = BEG_SERV_ADDR; rp < END_PROC_ADDR; rp++)
	if (rp->p_flags & PENDING) {
		m.m_type = KSIG;
		m.SIG_PROC = proc_number(rp);
		m.SIG_MAP = rp->p_pending;
		sig_procs--;
		if (lock_mini_send(proc_addr(HARDWARE), MM_PROC_NR, &m) != OK)
			panic("can't inform MM", NO_NUM);
		sigemptyset(&rp->p_pending); /* the ball is now in MM's court */
		rp->p_flags &= ~PENDING;/* remains inhibited by SIG_PENDING */
		lock_pick_proc();	/* avoid delay in scheduling MM */
		return;
	}
}


/*===========================================================================*
 *				umap					     *
 *===========================================================================*/
PUBLIC phys_bytes umap(rp, seg, vir_addr, bytes)
register struct proc *rp;	/* pointer to proc table entry for process */
int seg;			/* T, D, or S segment */
vir_bytes vir_addr;		/* virtual address in bytes within the seg */
vir_bytes bytes;		/* # of bytes to be copied */
{
/* Calculate the physical memory address for a given virtual address. */

  vir_clicks vc;		/* the virtual address in clicks */
  phys_bytes pa;		/* intermediate variables as phys_bytes */
#if (CHIP == INTEL)
  phys_bytes seg_base;
#endif

  /* If 'seg' is D it could really be S and vice versa.  T really means T.
   * If the virtual address falls in the gap,  it causes a problem. On the
   * 8088 it is probably a legal stack reference, since "stackfaults" are
   * not detected by the hardware.  On 8088s, the gap is called S and
   * accepted, but on other machines it is called D and rejected.
   * The Atari ST behaves like the 8088 in this respect.
   */

  if (bytes <= 0) return( (phys_bytes) 0);
  vc = (vir_addr + bytes - 1) >> CLICK_SHIFT;	/* last click of data */

#if (CHIP == INTEL) || (CHIP == M68000)
  if (seg != T)
	seg = (vc < rp->p_map[D].mem_vir + rp->p_map[D].mem_len ? D : S);
#else
  if (seg != T)
	seg = (vc < rp->p_map[S].mem_vir ? D : S);
#endif

  if((vir_addr>>CLICK_SHIFT) >= rp->p_map[seg].mem_vir+ rp->p_map[seg].mem_len)
	return( (phys_bytes) 0 );
#if (CHIP == INTEL)
  seg_base = (phys_bytes) rp->p_map[seg].mem_phys;
  seg_base = seg_base << CLICK_SHIFT;	/* segment origin in bytes */
#endif
  pa = (phys_bytes) vir_addr;
#if (CHIP != M68000)
  pa -= rp->p_map[seg].mem_vir << CLICK_SHIFT;
  return(seg_base + pa);
#endif
#if (CHIP == M68000)
#if (SHADOWING == 0)
  pa -= (phys_bytes)rp->p_map[seg].mem_vir << CLICK_SHIFT;
  pa += (phys_bytes)rp->p_map[seg].mem_phys << CLICK_SHIFT;
#else
  if (rp->p_shadow && seg != T) {
	pa -= (phys_bytes)rp->p_map[D].mem_phys << CLICK_SHIFT;
	pa += (phys_bytes)rp->p_shadow << CLICK_SHIFT;
  }
#endif
  return(pa);
#endif
}


/*==========================================================================*
 *				numap					    *
 *==========================================================================*/
PUBLIC phys_bytes numap(proc_nr, vir_addr, bytes)
int proc_nr;			/* process number to be mapped */
vir_bytes vir_addr;		/* virtual address in bytes within D seg */
vir_bytes bytes;		/* # of bytes required in segment  */
{
/* Do umap() starting from a process number instead of a pointer.  This
 * function is used by device drivers, so they need not know about the
 * process table.  To save time, there is no 'seg' parameter. The segment
 * is always D.
 */

  return(umap(proc_addr(proc_nr), D, vir_addr, bytes));
}


#if (CHIP == INTEL)
/*==========================================================================*
 *				alloc_segments				    *
 *==========================================================================*/
PUBLIC void alloc_segments(rp)
register struct proc *rp;
{
/* This is called only by do_newmap, but is broken out as a separate function
 * because so much is hardware-dependent.
 */

  phys_bytes code_bytes;
  phys_bytes data_bytes;
  int privilege;

  if (protected_mode) {
	data_bytes = (phys_bytes) (rp->p_map[S].mem_vir + rp->p_map[S].mem_len)
	             << CLICK_SHIFT;
	if (rp->p_map[T].mem_len == 0)
		code_bytes = data_bytes;	/* common I&D, poor protect */
	else
		code_bytes = (phys_bytes) rp->p_map[T].mem_len << CLICK_SHIFT;
	privilege = istaskp(rp) ? TASK_PRIVILEGE : USER_PRIVILEGE;
	init_codeseg(&rp->p_ldt[CS_LDT_INDEX],
		     (phys_bytes) rp->p_map[T].mem_phys << CLICK_SHIFT,
		     code_bytes, privilege);
	init_dataseg(&rp->p_ldt[DS_LDT_INDEX],
		     (phys_bytes) rp->p_map[D].mem_phys << CLICK_SHIFT,
		     data_bytes, privilege);
	rp->p_reg.cs = (CS_LDT_INDEX * DESC_SIZE) | TI | privilege;
#if _WORD_SIZE == 4
	rp->p_reg.gs =
	rp->p_reg.fs =
#endif
	rp->p_reg.ss =
	rp->p_reg.es =
	rp->p_reg.ds = (DS_LDT_INDEX*DESC_SIZE) | TI | privilege;
  } else {
	rp->p_reg.cs = click_to_hclick(rp->p_map[T].mem_phys);
	rp->p_reg.ss =
	rp->p_reg.es =
	rp->p_reg.ds = click_to_hclick(rp->p_map[D].mem_phys);
  }
}
#endif /* (CHIP == INTEL) */
\end{verbatim}

sys\_block.c

\begin{verbatim}
#include "syslib.h"

PUBLIC int sys_block(proc)
int proc;		/* process to block */
{
  message m;

  m.m1_i1 = proc;
  return(_taskcall(SYSTASK, SYS_BLOCK, &m));
}
\end{verbatim}

sys\_unblock.c

\begin{verbatim}
#include "syslib.h"

PUBLIC int sys_unblock(proc)
int proc;		/* process to unblock */
{
  message m;

  m.m1_i1 = proc;
  return(_taskcall(SYSTASK, SYS_UNBLOCK, &m));
}
\end{verbatim}

v\_sem.s

\begin{verbatim}
.sect .text
.extern __v_sem
.define _v_sem

.align 2

_v_sem:
	jmp __v_sem
\end{verbatim}

\_sem.c

\begin{verbatim}
#include <lib.h>
#include <minix/semaforo.h>
#include <minix/com.h>
#include <minix/constsemaf.h>
#include <stdio.h>
#include <string.h>

#define crear_sem	_crear_sem
#define p_sem		_p_sem
#define v_sem 		_v_sem
#define liberar_sem	_liberar_sem
#define inicializar _inicializar

semaforo crear_sem(char* nombre, int valor) {

	message m;
	semaforo new_sem;

	strcpy(m.NOMBRE_SEM,"");

	if(strlen(nombre) < M3_STRING) {
		strcpy(m.NOMBRE_SEM, nombre);
	} else {
		strncpy(m.NOMBRE_SEM, nombre, M3_STRING-1);
	}

	m.VALOR = valor;

	new_sem = _syscall(MM, CREAR_SEM, &m);
	if(new_sem==-1) {
		printf("error: crear_sem\n");
		return ERROR;
	} else {
		return new_sem;
	}

}

int p_sem(semaforo x) {

	message m;
	int r;

	m.SEMAFORO = x;

	r = _syscall(MM, P_SEM, &m);

	if(r==-1) {
		printf("error: p_sem\n");
		return ERROR;
	} else {
		return r;
	}

}

int v_sem(semaforo x) {

	message m;
	int r;

	m.SEMAFORO = x;

	r = _syscall(MM, V_SEM, &m);

	if(r==-1) {
		printf("error: v_sem\n");
		return ERROR;
	} else {
		return r;
	}

}

int liberar_sem(semaforo x) {

	message m;
	int r;

	m.SEMAFORO = x;

	r = _syscall(MM, LIBERAR_SEM, &m);

	if(r==-1) {
		printf("error: liberar_sem");
		return ERROR;
	} else {
		return r;
	}

}

void inicializar() {

	message m;

	_syscall(MM, INIT_ALL_SEM, &m);

	return 0;

}
\end{verbatim}



%%%%%%%%%%%%%%%%%%%%%%%%%%%%%%%%%%%%%%%%%%%%%%%%%%%%%%%%%%%%%%%%%%%%%%
% Referencias
%%%%%%%%%%%%%%%%%%%%%%%%%%%%%%%%%%%%%%%%%%%%%%%%%%%%%%%%%%%%%%%%%%%%%%
\clearpage
\section{Referencias}
\begin{itemize}
\item Sistemas Operativos - Dise'no e Implementaci'on (Andrew Tanenbaum - Prentice Hall 1998)
\item http://www.minix3.org/
\item http://es.wikipedia.org/wiki/Chmod
\end{itemize}
\end{document}
%%%
% EOF
%%%%%%%%%%%%%%%%%%%%%
