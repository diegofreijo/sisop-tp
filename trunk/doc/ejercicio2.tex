\subsection{Ejercicio 2: Herramientas}

Indique que hace el comando make y mknode. C'omo se utilizan 'estos comandos en la instalaci'on de MINIX y en la creaci'on de un nuevo kernel. Para el caso del make muestre un archivo ejemplo y explique que realiza cada uno de los comandos internos del archivo ejemplo.

El comando make sirve para compilar archivos. Make es un programa que es normalmente usado para desarrollar programas grandes que consisten en muchos archivos. Lleva cuenta de cuales archivos objeto dependen de cuales archivos fuentes y de cabecera. Cuando es ejecutado, hace la m'inima cantidad de recompilaciones para obtener el archivo destino actualizado.
El comando make se utiliza en la creaci'on de un nuevo kernel. Luego de modificar alg'un archivo fuente del kernel, uno debe reconstruir el kernel. En el directorio \textbf{/usr/src/tools} est'a el archivo Makefile, necesario para reconstruir el kernel.

El comando mknod sirve para crear archivos especiales ya sea archivos asociados a dispositivos de entrada/salida o bien directorios. El uso de 'este comando est'a reservado a los superusuarios.

A pesar de que en la instalaci'on de MINIX nosotros no tuvimos que, expl'icitamente, usar estos comandos, el comando mknod se utiliz'o para crear los nodos device en el nodo ra'iz: \textbf{/dev/tty}; \textbf{/dev/tty[0-2]}; \textbf{/dev/hd[0-9]}.
