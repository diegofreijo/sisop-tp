\subsection{Ejercicio 9}

Los dispositivos de E/S pueden dividirse en 2 categorias: dispositivos tipo 
bloque y dispositivos tipo caracter. Uno tipo bloque es el cual guarda 
infomaci'on en bloques de tama'no fijo, cada uno con su propia direcci'on. Los 
bloques comunes van del rango de los 128 bytes a los 1024 bytes. La propiedad 
esencial de este tipo de dispositivo es hacer posible la lectura o escritura de 
cada bloque independientemente de los otros. En otras palabas, instantaneamente 
el programa puede leer o escribir cualquiera de sus bloques. Un ejemplo de ellos 
son los discos.

Los dispositivos de E/S tipo caracter entrega o acepta una cadena de caracteres, 
sin respetar ninguna estructura de bloques. No es direccionable y no tiene 
ning'un tipo de operaci'on de b'usqueda. La terminal, impresoras, cintas de 
paepl, tarjetas perfordas, interfaces de red, mouses y muchos otros dipositivos 
no que no son como discos pueden verse como dispositivos tipo caracter.
Este modelo es en general suficiente para ser usado para hacer que el software 
del S.O. pueda trabajar con dispositivos de E/S independientes. El file system, 
por ejemplo, trabaja solo con dispositivos de bloques abstractos y deja la parte 
dependiente del dispositivo a software de bajo nivel llamado drivers de 
dispositivos.

El trabajo de los drivers de dispositivos es aceptar pedidos abstractos de un 
software independiente del dispositivo sobre 'el, y ver que el pedido sea 
realizado.

En MINIX, para cada clase de dispositivo de E/S, esta presente una tarea de E/S 
(I/O task) o driver de dispositivo.  Estos drivers son procesos terminados, cada 
uno con su propio estado, mapa de memoria y m'as. Estos driverse se comunican 
entre si, si es necesario,  con el file system usando el mecanismo de pasaje de 
mensajes estandar usado por todos los procesos de MINIX. Adem'as, cada driver de 
dispositivo esta escrito en un solo archivo fuente, como floppy.c o clock.c. La 
'unica diferencia entre los drivers y otros procesos es que los primeros estas 
linkeados con el kernel, y comparten todo el espacio de de direccionamiento.

El programa principal para cada driver de dispositivo tipo bloque es 
estructuralmente el mismo. Posee un ciclo infinito, el cual espera la llegada de 
un mensaje invocandolo. Si esto sucede, se llamar� a la operaci�n del tipo 
entregado por el mensaje.

La estructura es la siguiente:     

\begin{verbatim}

message mess;                       /* buffer del mensaje */

void io_task() {
    initialize();                   /* solo se inicializa una vez */
    while(TRUE){
        receive(ANY, &mess);        /* espera a un pedido de trabajo */
        caller = mess.source;       /* proceso desde donde vino el mensaje */
        switch(mess.type){
            case READ: rcode = dev_read(&mess);break;
            case WRITE: rcode = dev_write(&mess);break;
            /* Other cases go here, e.g., OPEN, CLOSE, IOTCTL */
            default: rcode = ERROR;
        }
        mess.type = TASK_REPLY;
        mess.status = rcode;        /* respuesta */
        send(caller, &mess);        /* Se envia respuesta al proceso llamador */
    }                                  
}

\end{verbatim}
                                    
Cuando el sistema inicia, cada uno de los drivers se inicializa para definir 
datos internos como tablas y similares cosas. Cada task intenta recibir un 
mensaje. Cuando alguno llega, se graba la identidad del llamador, y se ejecuta 
un procedimiento en particular para realizar el trabajo. Luego de finalizado, 
una respuesta es enviada al llamador, y el task vuelve a el tope del ciclo 
esperando un nuevo pedido.


Para generar un driver de dispositivo tipo bloque hay que hacer lo siguiente:

\begin{enumerate}
\item hay que alterar el /usr/include/minix/com.h. Este archivo es el que guarda el 
id de cada uno de los I/O task. Este id es llamado "major number" del 
dispositivo. Este n'umero especifica la clase de dispositivo, como los floppy, 
disco r'igido, o terminal. 
Todos los dispositivos con el mismo "major number" comparten el mismo codigo de 
driver dentro del S.O.

\item luego hay que alterar /usr/include/minix/config.h para determinar cuales son 
los dispositivos con los cuales el kernel va a ser compilado. Incluir todos 
estaria mal. Esto es para que sea m'as facil excluir el nuevo dispositivo. 

\item alterar el /usr/include/minix/const.h Se encarga de las constantes usadas por 
el MINIX. Se la usa para agregar en nuevo proceso a las NR\_TASKS. Setea el 
numero de las tareas predefinidas del kernel incluyendo los drivers de 
dispositivo. 

\item /usr/src/kernel/proto.h Hay que incluir el header o prototipo de la llamada 
al driver. Eso es, predefinir el punto de entrada del nuevo driver. Este punto 
es la primer funci'on que se ejecuta en el driver, como el main() es el punto de 
entrada de un archivo de C.
En el ejemplo, el task va a manejar un disco, se llamara en tal caso, 
\verb0disco_task0.

\begin{verbatim}
PUBLIC _PROTOTYPE( void disco_task, (void) );
\end{verbatim}

\item /usr/src/kernel/table.c En este archivo debe existir una entrada para las 
tasks. En ella se especifica el tama'no del stack que va a usar el nuevo driver. 

\begin{verbatim}
#define DISCO_STACK (4 * SMALL_STACK * ENABLE_DISCO)
\end{verbatim}

Adem'as se debe agregar dentro del struct tasktab una l'inea donde se relacionan 
el task(el nombred del driver), la pila y el programa del driver propiamente 
dicho en ese orden.

\begin{verbatim}
#if ENABLE_DISCO

{ disco_task, DISCO_STACK , "DISCO" },

#endif                 
\end{verbatim}

Donde \emph{DISCO} es el nombre del proceso.

\item \verb0/usr/src/kernel/table.c0 Tambi'en hay que agregar a este archivo una entrada 
para el nombre de la task en particular. Hay que agregar un archivo especial, en 
nuestro caso "DISCO" en /dev y ligar el proceso del nuevo driver con 'el. 
La nueva l'inea se agrega en el struct dmap. (como muesta se agrega el 
encabezado que aparece en el fuente ).

\begin{verbatim}
?    Open        Read/Write  Close       Task #      Device  File
-    ----        ----------  -----       -------     ------  ----

DT(1, dev_opcl,   call_task,  dev_opcl,   DISCO)      /* 7 = /dev/disco */
\end{verbatim}


\item Hay que generar el driver mismo. Depende de que es lo que maneje, la 
estructura del mismo. Para el ejemplo citado, el archivo se llamar� disco.c y se 
encontrar� en la ruta \verb0/usr/src/kernel/disco.c0

\item Es necesario alterar el Makefile del kernel para que el nuevo driver sea 
compilado con 'el.                        

\item Hay que generar el iNodo:

\verb0mknod /dev/disco Flags MayorNumber MinorNumber\verb0

\end{enumerate}
