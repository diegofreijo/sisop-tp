\subsection{Ejercicio 11}
Se realizaron los siguientes cambios

\verb�/usr/include/minix/callnr.h�

Se increment� en 1 el define NCALLS
Se agreg� \verb0#define LLAMSISTEMA0 con el nro correspondiente 

\verb0/usr/src/mm/table.c0

Nota: no se us� el nombre pedido x la c�tedra (newcall), ya que este fue usado para otras pruebas. Se realiz� una version del getpid desde MM y desde FS, sendas llamadas a sistema con nombres newcall y othercall respectivamente. 

El el mismo directorio el newcall.c y el othercall.c son source para probar dichas funciones
\begin{verbatim}
./newcall
./othercall
\end{verbatim}

Tienen la misma funcionalidad que \verb0./llam 10


\subsubsection{Pruebas}

En imagen minix para ejercicio 11

Fuente: \verb0/usr/ej11/llam.c0

Ejecutable: \verb0/usr/ej11/llam0

Modo de prueba

\begin{verbatim}
./llam [opcion]
\end{verbatim}

Opcion es del 1 al 5

\begin{enumerate}
\item Entrega el pid del programa
\item Entrega el pid del padre, que, al estar implementado en MM, es el pid del mm o sea 0
\item Entrega el puntero al segmento text, en hexadecimal
\item Entrega el puntero al segmento data, en hexadecimal
\item Entrega el puntero al segmento stack, en hexadecimal
\end{enumerate}

Para las opciones 3, 4 y 5, para verificar la correctitud de la misma se agreg� un ciclo while para evitar que el programa termine. Al ejecutar el programa con algunas de estas opciones, al presionar F2 veremos la informacion de los segmentos correspondientes. Para matar al proceso, abrimos otra consola, ejecutamos ps para ver el nro del mismo y con el comando \verb0kill [nro proc]0 lo eliminamos.
